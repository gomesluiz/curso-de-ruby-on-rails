
\begin{frame}[allowframebreaks, fragile,t]{Strings}
  \begin{itemize}
    \item Strings com aspas simples
    \begin{itemize}
      \item permitem a utilização de \alert{'} com $\backslash$
      \item mostra a string como foi escrita 
    \end{itemize}
    \item Strings com aspas duplas 
    \begin{itemize}
      \item interpreta caracteres especiais como $\backslash$n e $\backslash$t
      \item permite a interpolação de strings, evitando concatenação
    \end{itemize}
  \end{itemize}
  
  \lstinputlisting[style=RubyInputStyle, caption=strings.rb]{codigos/ruby/06-strings/strings.rb}
\pagebreak
  \begin{itemize}
    \item Métodos terminados com \alert{!} modificam a string
    \begin{itemize}
      \item a maioria retorna apenas um novo string
    \end{itemize}
    \item Permite o uso do \%Q\{textos longos com multiplas linhas\}
    \begin{itemize}
      \item o mesmo comportamento de strings com aspas duplas
    \end{itemize} 
    \item \alert{É essencial dominar a API de Strings do Ruby}
  \end{itemize}  
  
  \lstinputlisting[style=RubyInputStyle, caption=more\_strings.rb]{codigos/ruby/06-strings/more_strings.rb} 
\end{frame}

\begin{frame}[fragile,t]{Símbolos}
  \begin{itemize}
    \item \alert{:simbolo} $-$ string altamente otimizadas
    \begin{itemize}
      \item ex. :domingo, :dolar, :calcio, :id
    \end{itemize}
    \item Constantes que não precisam ser pré-declaradas
    \item Garantia de \alert{unicidade} e \alert{imutabilidade}
    \item Podem ser convertidos para uma \alert{String} com \alert{to\_s} 
    \begin{itemize}
      \item ou de \alert{String} para \alert{Símbolo} com \alert{to\_sym}
    \end{itemize}
  \end{itemize}   
\end{frame}

\begin{frame}[fragile,t]{Recapitulando}
  \begin{itemize}
    \item A interpolação evita a concatenação de strings
    \item Strings oferecem uma API muito útil
  \end{itemize}
\end{frame}



