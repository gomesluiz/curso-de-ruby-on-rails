\documentclass[t, 				             
			   final,
			   12pt, 				         
			   xcolor={usenames,dvipsnames}, 
			   table]{beamer}

% pacotes utilizados.
\usepackage[alf]{abntex2cite}
\usepackage{amsmath}
\usepackage[brazil]{babel}
\usepackage{booktabs}
\usepackage{caption}
\usepackage[utf8]{inputenc}
\usepackage{listings}
\usepackage{multicol}
\usepackage{multirow}
\usepackage{todo}


% configuração do tema
\usetheme[pageofpages=de,
          bullet=square,			
          titleline=true,				
          alternativetitlepage=true,			
          titlepagelogo=imagens/logo-puc,	
          watermarkheight=70px,		
          watermarkheightmult=4	
          ]{Torino}

\setbeamertemplate{sections/subsections in toc}[square]
\setbeamertemplate{bibliography item}[default]

\usecolortheme{freewilly}


\definecolor{javared}{rgb}{0.6,0,0} % for strings
\definecolor{javagreen}{rgb}{0.25,0.5,0.35} % comments
\definecolor{javapurple}{rgb}{0.5,0,0.35} % keywords
\definecolor{javadocblue}{rgb}{0.25,0.35,0.75} % javadoc
 
\lstset{}

\lstdefinestyle{BashInputBasicStyle}{
	language=bash,
	basicstyle=\normalsize\ttfamily,
	columns=fullflexible,
	tabsize=2,
	showstringspaces=false,
	frame=single,
	inputencoding=utf8,
	rulecolor=\color{gray}
}

\lstdefinestyle{BashInputStyle}{
  language=bash,
  basicstyle=\normalsize\ttfamily,
  numbers=left,
  numberstyle=\tiny,
  numbersep=2pt,
  frame=tb,
  columns=fullflexible,
  tabsize=2,
  showstringspaces=false,
  commentstyle=\color{gray},
  inputencoding=utf8,
  rulecolor=\color{gray}
}

\lstdefinestyle{RubyInputStyle}{
    language=ruby,
    basicstyle=\scriptsize\ttfamily,
    keywordstyle=\color{javapurple},
    identifierstyle=\color{black},
    commentstyle=\color{javagreen},
	stringstyle=\color{blue},
    showstringspaces=false,
    numbers=left,
    numberstyle=\color{gray}\tiny,
    tabsize=3,
    extendedchars=\true,
    inputencoding=utf8,
%   frame=single, 
    columns=fixed,
    backgroundcolor=\color{red!32!green!33!blue!5}
}    
%  language=ruby,
%  basicstyle=\normalsize\ttfamily,
%  keywordstyle=\color{OrangeRed},
%  identifierstyle=\color{Turquoise},
%  commentstyle=\color{gray},
%  stringstyle=\color{YellowOrange},
%  numbers=left,
%  numberstyle=\tiny,
%  numbersep=2pt,
%  frame=tb,
%  columns=fullflexible,
%  backgroundcolor=\color{white!80},
%  linewidth=0.9\linewidth,
%  tabsize=2,
%  showstringspaces=false
%  inputencoding=utf8


\lstdefinestyle{JavaInputStyle}{
	language=Java,
	basicstyle=\ttfamily,
	keywordstyle=\color{javapurple}\bfseries,
	stringstyle=\color{javared},
	commentstyle=\color{javagreen},
	morecomment=[s][\color{javadocblue}]{/**}{*/},
	numbers=left,
	numberstyle=\tiny\color{black},
	numbersep=10pt,
	tabsize=2,
	showspaces=false,
	showstringspaces=false,
	frame=tb,
	columns=fullflexible,
	backgroundcolor=\color{white!80},
	linewidth=0.9\linewidth,
	inputencoding=utf8
}

\begin{document}
	\author{Luiz Alberto Ferreira Gomes}
\title{Ruby On Rails}
\subtitle{Seminários da Computação}
\institute{Curso de Ciência da Computação}
\date{\today}

	\include{Capa}
	\include{Agenda}  	
    
    \section{Ruby on Rails}
	%-------------------------------------------------------------------------------------- Início
\begin{frame}[fragile, plain, c]{Rails}
	\begin{center}
		\large Rails é um \alert{framework} para construção de \alert{aplicações web} baseado na \alert{linguagem Ruby}.
	\end{center}
\end{frame}
%-------------------------------------------------------------------------------------- Início
\begin{frame}[fragile, t]{Rails}
	\begin{itemize}
		\item Uma \alert{gem} (pacote) escrita em Ruby
		\item 100\% \alert{open-source}, MIT License
		\item Fornece \alert{geradores} de código e scripts de \alert{automação} de testes
		\item Ferramentas adicionais são fornecidas no ecossistema Rails:
	\end{itemize}
\end{frame}

%-------------------------------------------------------------------------------------- Início
\begin{frame}[fragile, t]{Ferramentas Adicionais}
	\begin{itemize}
		\item \alert{Rake} - utilitário similar ao \textbf{make do Unix} para criar e migrar bancos de dados, limpar sessões de uma Web app
		\item \alert{Puma} - servidor web de desenvolvimento para execução de aplicações Rails
		\item \alert{SQLite} - um servidor de banco de dados simples pré-instalado como o Rails
		\item \alert{Rack Middleware} - interface padronizado para interação entre um servidor web e uma Web App
	\end{itemize}
\end{frame}	
	%%\section{Histórico de Evolução}
%-------------------------------------------------------------------------------------- Início
\begin{frame}[allowframebreaks,fragile,t]{Histórico do Rails}
  \begin{itemize}
  	\item Rails é um \textit{framework} para construção de aplicações web
    \item David Heinemeier Hanson \alert{derivou} o Rails a partir do BaseCamp --
      uma ferramenta de gestão de projetos da empresa 37Signals.
    \begin{itemize}
	\item a primeira versão de código aberto (em inglês: \textit{open source})foi liberada em 
	  julho de 2004.
	\item mas direitos para que outros desenvolvedores \alert{colaborassem} com o projeto foram liberadosw
	  em fevereiro de 2005.
    \end{itemize}
    \item Em agosto de 2006, o Ruby on Rails atingiu um \alert{marco importante} quando a Apple dicidiu
      distribuído juntamente com a versão do seu sistema operacional Mac OS X v10.5 "Leopard"
    \begin{itemize}
     \item nesse mesmo no o Rails começou a ganhar muita atenção da comunidade de desenvolvimento web.
    \end{itemize}
    \item Rails é utilizado por diversas companhias, como por exemplo:
    \begin{itemize}
     \item Airbnb, BaseCamp, Disney, GitHub, Hulu, Kickstarter, Shopify e Twitter.
    \end{itemize}

  \end{itemize}

  \begin{figure}[h]
    \includegraphics[scale=0.3]{imagens/rails-history.png}  
  \end{figure}
\end{frame}

	\begin{frame}[allowframebreaks,fragile,t]{Filosofia do Rails}
  \begin{itemize}
    \item Ruby on Rails é 100\% open-source, disponível por meio da MIT License:
      \url{http://opensource.org/licenses/mit-license.php}.  
    \item \alert{Convenção} acima da Configuração (em inglês: \textit{Convention over Configuration} (CoC))
    \begin{itemize}
	\item se nomeação segue certas convenções, não há necessidade de arquivos de configuração.
	\begin{exampleblock}{Exemplo:}
	  \begin{verbatim}
      FilmesController#show -> filmes_controler.rb 
      FilmesController#show -> views/filmes/show.html.er
    \end{verbatim}
	\end{exampleblock}
    \end{itemize}
    \item \alert{"Don't Repeat Yourself"} (DRY) sugere que escrever que o mesmo código várias 
      vezes é uma coisa ruim
    
    \item O \textit{Representational State Transfer} (REST) é o melhor padrão para desenvolvimento de aplicações web
    \begin{itemize}
     \item organiza a sua aplicação em torno de \alert{recursos} e \alert{padrões} HTTP (verbs)
    \end{itemize}
  \end{itemize}   
\end{frame}

	%%\section{MVC em Ação no Rails}
%%-------------------------------------------------------------------------------------- Início
\begin{frame}[t, fragile]{Model-View-Controller}
	\begin{itemize}
		\item O framework Rails é contruído em cima do Design Pattern Model View Controller(MVC):
	\end{itemize}
	\begin{figure}[h!]
		\centering
		\includegraphics[width=0.70\textwidth]{imagens/mvc.jpg}
	\end{figure}
\end{frame}
 

    
    \section{Metodologia de Trabalho}
    \begin{frame}[fragile,t]{Metodologia de Trabalho}
	\begin{figure}[h!]
		\centering
		\includegraphics[width=.9\textwidth]{imagens/metodologia-de-trabalho.jpg}
	\end{figure}
\end{frame}

\begin{frame}[allowframebreaks, fragile,t]{Passos Iniciais do Blog App}
	\begin{enumerate}
	    \item Inicie uma janela de terminal e digite no prompt:
	     \begin{lstlisting}[style=BashInputBasicStyle]
		     $ cd 
		     $ rails new blog
	     \end{lstlisting}
%	    \item Utilize o gerador \verb|scaffold| para criar os 
%	    componentes MVC para as postagens e os comentários
%	     \begin{lstlisting}[style=BashInputBasicStyle]
%					$ rails generate scaffold post \ 
%			     title:string body:text
%					$ rails generate scaffold comment post_id:integer \
%					 body:text
%	     \end{lstlisting}
%	    \item Gere as tabelas \verb|post| e \verb|comment| no banco de dados
%	    \begin{lstlisting}[style=BashInputBasicStyle]
%		    $ rake db:migrate
%	    \end{lstlisting}
%	    
%	    \item Visualize todas as URLs reconhecidas pela sua aplicação digitando:
%	    \begin{lstlisting}[style=BashInputBasicStyle]
%	    $ rake routes
%	    \end{lstlisting}
	     \item Inicie o servidor web embutido:
	     \begin{lstlisting}[style=BashInputBasicStyle]
	     $ rails s
	     \end{lstlisting}
	     
	     \item Abra uma janela do navegador e digite:
	     \begin{lstlisting}[style=BashInputBasicStyle]
	     $ http://localhost:3000
	     \end{lstlisting}
	\end{enumerate}
  
\end{frame}


    \section{Esqueleto da Aplicação}
    
\begin{frame}[allowframebreaks,fragile,t]{Estrutura de uma Aplicação Rails}
  \begin{figure}[h]
    \includegraphics[scale=0.35]{imagens/rails-esqueleto-01.png}  
  \end{figure}
  \begin{figure}[h]
    \includegraphics[scale=0.35]{imagens/rails-esqueleto-02.png}  
  \end{figure}
  \begin{figure}[h]
    \includegraphics[scale=0.35]{imagens/rails-esqueleto-03.png}  
  \end{figure}
\end{frame}
   
    \section{Primeira Aplicação}
    %%-------------------------------------------------------------------------------------- Início
\begin{frame}[allowframebreaks, fragile,t]{Hora de Colocar a Mão na Massa}
	\begin{enumerate}
		\item Inicie uma janela de terminal e digite no prompt:
			\begin{lstlisting}[style=BashInputBasicStyle]
			$ rails new blog
			\end{lstlisting}

		\item Mude para o diretório da aplicao (RAILS.root)
			\begin{lstlisting}[style=BashInputBasicStyle]
			$ cd blog
			\end{lstlisting}

		\item Execute o servidor web embutido:
		\begin{lstlisting}[style=BashInputBasicStyle]
			$ rails server
		\end{lstlisting}
		
		\item Abra uma janela do navegador e digite:
		\begin{lstlisting}[style=BashInputBasicStyle]
			http://localhost:3000
		\end{lstlisting}

		\item Verifique o conteúdo do arquivo de configuração \verb|database.yml|:
		\begin{lstlisting}[style=BashInputBasicStyle]
			$ cat config/database.yml
		\end{lstlisting}

		\item Crie o banco de dados de desenvolvimento e testes:
		\begin{lstlisting}[style=BashInputBasicStyle]
			$ rake db:create
		\end{lstlisting}

		\item Crie o modelo Post:
		\begin{lstlisting}[style=BashInputBasicStyle]
			$ rails g model Post title:string body:text
		\end{lstlisting}

		\item Implemente modelo Post no banco de dados com {\it migrations}:
		\begin{lstlisting}[style=BashInputBasicStyle]
			$ rake db:migrate
		\end{lstlisting}

		\item Crie o controlador PostsController:
		\begin{lstlisting}[style=BashInputBasicStyle]
			$ rails g controller Posts 		
		\end{lstlisting}

		\item Modifique o arquivo \verb|config/routes.rb| para acrescentar a rota para
		os posts:
		\begin{lstlisting}[style=RubyInputStyle]
			Rails.application.routes.draw do 
				resources :posts
			end 
		\end{lstlisting}	
		
		\item Execute o comando \verb!rake! para visualizar as rotas para os
		posts:
		\begin{lstlisting}[style=BashInputBasicStyle]
			$ rake routes
		\end{lstlisting}
		
		\item Reinicie o servidor web e acesse a url \url{http:\\localhost:3000/posts/new}. Veja o erro que ocorreu.

	\end{enumerate}
\end{frame}
    \include{rails-rest}
    %%-------------------------------------------------------------------------------------- Início
\begin{frame}[allowframebreaks, t, fragile]{Ação: New}
	\begin{figure}[h!]
		\centering
		\includegraphics[width=0.75\textwidth]{imagens/mvc-action-new.jpg}
	\end{figure}
	\framebreak
	\begin{itemize}
		\item Um novo objeto \alert{@post} da classe \alert{Post} é instanciado 
		\item Procura pela visão \alert{new.html.erb} para renderizar a resposta
		\begin{lstlisting}[style=RubyInputStyle, caption=app/controllers/posts\_controller.rb]
Class PostsController < ApplicationController
  def new
    @post = Post.new 
  end 
end
		\end{lstlisting}
	\end{itemize}
\end{frame}

\begin{frame}[allowframebreaks, t, fragile]{Visão: New}
	\begin{itemize}
		\item Reinicie o servidor web e acesse a url \url{http:\\localhost:3000/posts/new}. Veja o erro que ocorreu.
		\item Implemente a visão \alert{new.html.erb}:
		\begin{lstlisting}[style=RubyInputStyle, caption=views/posts/new.html.erb]
<h1>Novo Post</h1>
<%= form_with scope: :post, url: posts_path, 
	local: true do |form| %>
	<p>
		<%= form.label :title %><br>
		<%= form.text_field :title %>
	</p>
	
	<p>
		<%= form.label :body %><br>
		<%= form.text_area :body %>
	</p>
	
	<p>
		<%= form.submit %>
	</p>
<% end %>	
		\end{lstlisting}
	\end{itemize}	
\end{frame}
    %%-------------------------------------------------------------------------------------- Início
\begin{frame}[allowframebreaks, t, fragile]{Ação: Create}
  \begin{figure}[h!]
		\centering
		\includegraphics[width=0.75\textwidth]{imagens/mvc-action-create.jpg}
	\end{figure}
	\framebreak
  \begin{itemize}
		\item Um novo objeto \alert{@post} da classe \alert{Post} é criado com os parâmetros que foram passados pelo 
			formulário \alert{new}
		\item Tenta \alert{salvar} o objeto \alert{@post} no \alert{banco de dados}
%		\item Se sucesso, redireciona para o template \alert{show}
%		\item Se insucesso, renderiza o template \alert{new} novamente
		\begin{lstlisting}[style=RubyInputStyle, caption=controllers/posts\_controller.rb]
class PostsController < ApplicationController
  def new
      @post = Post.new
  end

  def create 
      @post = Post.new(post_params)
      
      @post.save
      redirect_to @post
  end

private 
  def post_params 
    params.require(:post).permit(:title, :body)
  end
end          
		\end{lstlisting}		
		\begin{itemize}
			\item a linha 15 implementa \alert{strong parameters} para aumentar a segurança da aplicação
    \end{itemize}
    \item Como a ação {\bf Show} ainda não foi implementada, ocorrerá uma erro quando
    o botão {\bf Submit} for pressionado.
	\end{itemize}	
\end{frame}
    %%-------------------------------------------------------------------------------------- Início
\begin{frame}[allowframebreaks, t, fragile]{Ação: Show}
	\begin{figure}[h!]
		\centering
		\includegraphics[width=0.75\textwidth]{imagens/mvc-action-show.jpg}
	\end{figure}
	\framebreak
	\begin{itemize}
		\item Recupera \alert{uma} postagem específica no parâmetro \alert{id} passado como parte da URL
		\item (Implicitamente) procura pelo \alert{show.html.erb} para renderizar a resposta
		\begin{lstlisting}[style=RubyInputStyle, caption=controllers/posts\_controller.rb]
class PostsController < ApplicationController
	def show 
		@post = Post.find(params[:id])
	end

	def new
		@post = Post.new
	end

	def create 
		@post = Post.new(post_params)
		
		@post.save
		redirect_to @post
	end
private 
	def post_params 
		params.require(:post).permit(:title, :body)
	end
end
		\end{lstlisting}
	\end{itemize}	
\end{frame}

%%-------------------------------------------------------------------------------------- Início
\begin{frame}[allowframebreaks, t, fragile]{Visão: Show}
	\begin{itemize}
		\item Implemente a visão \alert{show.html.erb}:
		\begin{lstlisting}[style=RubyInputStyle, caption=views/posts/show.html.erb]
<p>
	<strong>Title:</strong>
	<%= @post.title %>
</p>
<p>
<strong>Body:</strong>
	<%= @post.body %>
</p>
		\end{lstlisting}		
	\end{itemize}	
\end{frame}
    
\begin{frame}[t, fragile]{Database Console}
	\begin{itemize}
		\item O comando \alert{rails db} fornece uma console para acesso aos bancos dados
		MySQL, PostgreSQL e SQLite.
	\end{itemize}
	
	\begin{lstlisting}[style=BashInputBasicStyle, basicstyle=\tiny\ttfamily,  keepspaces=true]
		$ rails db
		SQLite version 3.8.7.1 2014-10-29 13:59:56
		Enter ".help" for usage hints.
		sqlite> .headers on
		sqlite> .mode columns
		sqlite> select * from posts;
		id          title             body            created_at                  updated_at                
		----------  ----------------  --------------  --------------------------  --------------------------
		5           A Linguagem Ruby  Ruby e legal.  2016-04-30 22:45:20.636363  2016-04-30 22:45:20.636363
		sqlite> 
	\end{lstlisting}
	
	\begin{itemize}
		\item Dica: utilize \alert{headers on} e \alert{mode coluns}
	\end{itemize}
	
\end{frame}
    \include{rails-rails-console}

    \section{Versionando a Primeira Aplicação}
    %-------------------------------------------------------------------------------------- Início
\begin{frame}[allowframebreaks, t, fragile]{Controle Automatizado de Versão}
	\begin{itemize}
		\item GitHub
        \item git
        \item git config
		\item git init
		\item git add 
		\item git commit
		\item git remote
		\item git push
	\end{itemize}
\end{frame}
%-------------------------------------------------------------------------------------- Início
\begin{frame}[fragile, plain, c]{GitHub}
    \begin{center}
        \begin{figure}[h]
            \includegraphics[width=0.5\textwidth]{imagens/github.png}
            \caption{www.github.com}
        \end{figure}
	\end{center}
\end{frame}
%-------------------------------------------------------------------------------------- Início
\begin{frame}[fragile, plain, c]{git}
    \begin{center}
        \begin{figure}[h]
            \includegraphics[width=0.5\textwidth]{imagens/git.png}
            \caption{https://git-scm.com/}
        \end{figure}
	\end{center}
\end{frame}
%-------------------------------------------------------------------------------------- Início
\begin{frame}[fragile, plain, c]{git $+$ GitHub}
    \begin{center}
        \begin{figure}[h]
            \includegraphics[width=0.5\textwidth]{imagens/git_github.jpeg}
            \caption{https://git-scm.com/}
        \end{figure}
	\end{center}
\end{frame}
    
%%-------------------------------------------------------------------------------------- Início
\begin{frame}[allowframebreaks, fragile,t]{Hora de Colocar a Mão na Massa}
	\begin{enumerate}
        \item Reinicie o repositório local 
		\begin{lstlisting}[style=BashInputBasicStyle]
			$ git init
		\end{lstlisting}

        \item Efetive as mudanças realizadas no repositório local:  		
        \begin{lstlisting}[style=BashInputBasicStyle]
			$ git add .
			$ git commit -m "commit inicial"
		\end{lstlisting}

		\item Registre as alterações no repositório remoto:
        \begin{lstlisting}[style=BashInputBasicStyle]
$ git branch -M main
$ git remote add origin https://endereco/do/seu/repositorio.git
$ git push -u origin main
		\end{lstlisting}
	\end{enumerate}
\end{frame}

%    echo "# blog" >> README.md
%    git init
%    git add README.md
%    git commit -m "first commit"
%    git remote add origin https://github.com/gomesluiz/blog.git
%    git push -u origin master
    
    
    \section{Para Saber Mais}
	%%-------------------------------------------------------------------------------------- Início
\begin{frame}[fragile,t]{Para Saber Mais}
  \begin{itemize}
    \item \url{https://www.ruby-lang.org/en/}
    \begin{itemize}
     \item referência oficial da linguagem Ruby onde a toda a sua documentação está disponível
	para ser consultada.
    \end{itemize}

    \item \url{http://rubyonrails.org/}
    \begin{itemize}
     \item referência oficial do framework Rails onde a toda a sua documentação está disponível
	para ser consultada.
    \end{itemize}
    
    \item \url{http://www.codecademy.com/pt/tracks/ruby}
    \begin{itemize}
     \item curso iterativo em portugês sobre a linguagem Ruby.
    \end{itemize}

	\item \url{https://gorails.com/setup/ubuntu/16.04}
	\begin{itemize}
		\item guia para instalação do Ruby on Rails no Ubuntu e no Mac OSX.
	\end{itemize}
  \end{itemize}
  
  
\end{frame}
\end{document}
