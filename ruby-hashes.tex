
\begin{frame}[allowframebreaks, fragile,t]{Hashes}
  \begin{itemize}
    \item \alert{Coleção indexada} de objetos
    \item Criados com \{\} ou \alert{Hash.new}
    \item Também conhecidos como \alert{arrays associativos}
    \item Pode ser indexado com \alert{qualquer} tipo de dados
    \begin{itemize}
      \item não apenas com \alert{inteiros}
    \end{itemize}
    \item Acessados utilizando o operador \alert{[ ]}
    \item Atribuição de valores poder feita usando:
    \begin{itemize}
      \item \alert{$=>$} (criação)
      \item \alert{$[  ]$} (pós-criação)
    \end{itemize}
  \end{itemize}
  
  \lstinputlisting[style=RubyInputStyle, caption=hashes.rb]{codigos/ruby/09-hashes/hashes.rb}
\pagebreak
  \begin{itemize}
    \item E se tentarmos \alert{acessar} um valor em Hash que \alert{não existe}?
    \begin{itemize}
      \item \alert{nil} é retornado
    \end{itemize}
    \item Se o Hash é criado com \alert{Hash.new(0)} 0 é retornado.
  \end{itemize}  
  
  \lstinputlisting[style=RubyInputStyle, caption=word\_frequency.rb]{codigos/ruby/09-hashes/word_frequency.rb}
  
\pagebreak
  \begin{itemize}
    \item A partir da versão 1.9
    \begin{itemize}
      \item A ordem de criação do Hash é \alert{mantida}
      \item A sintaxe \alert{simbolo:} pode ser utilizada, se símbolos são utilizados como chave
      \item Se o Hash é o \alert{último argumento}, \{\} são opcionais
    \end{itemize}
  \end{itemize}  
  
  \lstinputlisting[style=RubyInputStyle, caption=more\_hashes.rb]{codigos/ruby/09-hashes/more_hashes.rb}

\end{frame}

%-------------------------------------------------------------------------------------- Início
\begin{frame}[fragile,t]{Exercícios}
  \begin{enumerate}
    \item Escreva os \textit{script} em Ruby e verifique os resultados após a sua execução.
  \end{enumerate}
	\begin{lstlisting}[style=RubyInputStyle]
opostos = {positivo: "negativo"
, aberto: "fechado"
, direita: "esquerda"}
opostos.each_key { |key| puts key }
opostos.each_value { |value| puts value }
opostos.each { |key, value| puts "O oposto de #{key} eh #{value}" } 
	\end{lstlisting}
\end{frame}

\begin{frame}[fragile,t]{Recapitulando}
  \begin{itemize}
    \item Hashes são \alert{coleções indexadas}
    \item Usado de forma \alert{similar aos arrays}
  \end{itemize}
\end{frame}



