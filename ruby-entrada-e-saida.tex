%-------------------------------------------------------------------------------------- Início
\begin{frame}[fragile,t]{Entrada pelo Teclado}
    \begin{itemize}
      \item \verb!gets! é método \alert{padrão para receber} um valor pelo teclado 
          \begin{lstlisting}[style=RubyInputStyle]
  # recebe um valor do tipo string.
  nome = gets
        \end{lstlisting}
      \item Utilize \verb!gets.chomp! para remover o caracter de nova linha.
          \begin{lstlisting}[style=RubyInputStyle]
  # remove o caracter de nova linha.
  nome = gets.chomp
        \end{lstlisting}
        \item Utilize \verb!gets.chomp.to_i! para converter o valor lido para inteiro. 
          \begin{lstlisting}[style=RubyInputStyle]
  # converte a string recebida para inteiro.
  idade = gets.chomp.to_i
        \end{lstlisting}
    \end{itemize}   
  \end{frame}
%-------------------------------------------------------------------------------------- Início
\begin{frame}[fragile,t]{Saída na Tela}
    \begin{itemize}
      \item \verb!puts! é método \alert{padrão} para impressão em tela 
      \begin{itemize}
          \item insere uma quebra de linha após a impressão
          \item similar ao \verb!System.out.println! do Java
      \end{itemize}
      \begin{lstlisting}[style=RubyInputStyle]
# exibe da tela do computador.
puts "Informacoes do jogador"
puts "Nome  %s" % nome 
puts "Idade %d" % idade
puts "Nome %s \nIdade %d" % [nome, idade]
      \end{lstlisting}
    \end{itemize}   
  \end{frame}

\begin{frame}[allowframebreaks,fragile,t]{Hora de Colocar as Mão na Massa}
  \begin{enumerate}
    \item Escreva um \textit{script} Ruby leia o nome de um jogador do jogo de adivinha é um 
    palpite, ambos digitados pelo teclado. Após isto, o script deverá apresentar os valores lidos.
  \end{enumerate}
  \framebreak
\end{frame}
  