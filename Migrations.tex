\begin{frame}[allowframebreaks, t, fragile]{Migrations}
	
		
	\begin{itemize}
		\item \alert{Como podemos rastrear e desfazer alterações em um banco de dados?}
		\item Não existe uma maneira fácil - manualmente é confuso e propenso a erros.
		\item Tipicamente, comandos SQL são dados para criar e modificar tabelas em um
		banco de dados
		\item Mas se houver a necessidade de trocar o banco de dados "durante o voo"?
		\begin{itemize}
		  \item por exemplo, desenvolve-se em SQLite e implanta-se em MySQL.
		\end{itemize}
	\end{itemize}
	
	\begin{center}
		\textcolor{Turquoise}{{\huge SOLUÇÃO: Migrations}}
	\end{center}
	
	\framebreak
	
	\begin{itemize}
		\item A cada vez que o \alert{scaffold} é executado na aplicação, o Rails cria um
		arquivo de \alert{migration} de banco de dados. Este arquivo é armazenado em \alert{db/migrate}
		\item Por exemplo: o arquivo \verb|20160430140114_create_posts.rb|
	\end{itemize}
	  
	\lstinputlisting[style=RubyInputStyle]{codigos/blog_1/db/migrate/20160430140114_create_posts.rb}
	
	\begin{itemize}
		\item Rails utiliza o comando \alert{rake} para executar os \alert{migrations} e fazer as alterações
		no banco de dados.
	\end{itemize}
	
	\begin{lstlisting}[style=BashInputBasicStyle]
		$ rake db:migrate
	\end{lstlisting}
	
\end{frame}