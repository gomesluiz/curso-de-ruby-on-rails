\begin{frame}[allowframebreaks, t, fragile]{Prática 2}
	\begin{itemize}
		\item Inicialize \alert{na pasta da aplicação} a console do banco de dados e
		configure a sua exibição:
		\begin{lstlisting}[style=BashInputBasicStyle]
			$ rails db
			sqlite> .headers on
			sqlite> .mode columns
		\end{lstlisting}
		
		\item Exiba os colunas da tabela \verb|posts|:
		\begin{lstlisting}[style=BashInputBasicStyle]
			sqlite> .schema posts
		\end{lstlisting}
		
		\framebreak
		\item Exiba todos os \verb|posts|:
		\begin{lstlisting}[style=BashInputBasicStyle]
			sqlite> SELECT * FROM posts;
		\end{lstlisting}
		
		\item Exiba todos os \verb|posts| ordenados pelo título (title):
		\begin{lstlisting}[style=BashInputBasicStyle]
			sqlite> SELECT * FROM posts ORDER BY title;
		\end{lstlisting}
		
		\item Exiba um \verb|post|:
		\begin{lstlisting}[style=BashInputBasicStyle]
			sqlite> SELECT * FROM posts LIMIT 1
		\end{lstlisting}
		
		\item Exiba o \verb|post| cujo \verb|id| é 2:
		\begin{lstlisting}[style=BashInputBasicStyle]
			sqlite> SELECT * FROM posts WHERE id=2;
		\end{lstlisting}
	\end{itemize}
\end{frame}