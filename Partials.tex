%%-------------------------------------------------------------------------------------- Início
\begin{frame}[allowframebreaks, t, fragile]{Partials}
	\begin{itemize}
		\item Rails encoraja o princípio DRY	
		\item O laioute da aplicação é mantida em um único local no arquivo \alert{application.html.erb}
		\item O código comum dos templates ser reutilizado em \alert{múltiplos templates}
		\item Por exemplo, os formulários do \alert{edit} e do \alert{new} - são realmente muito diferentes ?
		\item Partials são similares aos templates regulares, mas ele possuem capacidades mais \alert{refinadas}
		\item Nomes de partials começam com \alert{underscore} (\_) 
		\item Partials são renderizados com \alert{render 'partialname'} (sem underscore)
		\item \alert{render} também aceita um segundo argumento, um hash com as variáveis locais utilizadas no partial
		\item Similar a passagem de variáveis locais, o \alert{render} pode receber um objeto
		\item \alert{<%= render @post %>} renderizara \alert{app/views/posts/_posts.html.erb} com o conteúdo da variavel @post
		\item \alert{<%= render @posts %>} renderiza uma coleção e é equivalente a:
		\begin{lstlisting}[style=RubyInputStyle, caption=posts_controller.rb]
			<% @posts.each do |posts	| %> 
				<%= render post %>
			<% end %>
		\end{lstlisting}		
		\begin{itemize}
			\item \alert{_form.html.erb}
		\end{itemize}
				
	\end{itemize}	
\end{frame}


