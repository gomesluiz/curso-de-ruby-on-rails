
\begin{frame}[t, fragile]{Database Console}
	\begin{itemize}
		\item O comando \alert{rails db} fornece uma console para acesso aos bancos dados
		MySQL, PostgreSQL e SQLite.
	\end{itemize}
	
	\begin{lstlisting}[style=BashInputBasicStyle, basicstyle=\tiny\ttfamily,  keepspaces=true]
		$ rails db
		SQLite version 3.8.7.1 2014-10-29 13:59:56
		Enter ".help" for usage hints.
		sqlite> .headers on
		sqlite> .mode columns
		sqlite> select * from posts;
		id          title             body            created_at                  updated_at                
		----------  ----------------  --------------  --------------------------  --------------------------
		5           A Linguagem Ruby  Ruby e legal.  2016-04-30 22:45:20.636363  2016-04-30 22:45:20.636363
		sqlite> 
	\end{lstlisting}
	
	\begin{itemize}
		\item Dica: utilize \alert{headers on} e \alert{mode coluns}
	\end{itemize}
	
\end{frame}

\begin{frame}[allowframebreaks, t, fragile]{Hora de Colocar a Mão na Massa}
	\begin{itemize}
		\item Inicialize \alert{na pasta da aplicação} a console do banco de dados e
		configure a sua exibição:
		\begin{lstlisting}[style=BashInputBasicStyle]
			$ rails db
			sqlite> .headers on
			sqlite> .mode columns
		\end{lstlisting}
		
		\item Exiba os colunas da tabela \verb|posts|:
		\begin{lstlisting}[style=BashInputBasicStyle]
			sqlite> .schema posts
		\end{lstlisting}
		
		\framebreak
		\item Exiba todos os \verb|posts|:
		\begin{lstlisting}[style=BashInputBasicStyle]
			sqlite> SELECT * FROM posts;
		\end{lstlisting}
		
		\item Exiba todos os \verb|posts| ordenados pelo título (title):
		\begin{lstlisting}[style=BashInputBasicStyle]
			sqlite> SELECT * FROM posts ORDER BY title;
		\end{lstlisting}
		
		\item Exiba um \verb|post|:
		\begin{lstlisting}[style=BashInputBasicStyle]
			sqlite> SELECT * FROM posts LIMIT 1
		\end{lstlisting}
		
		\item Exiba o \verb|post| cujo \verb|id| é 2:
		\begin{lstlisting}[style=BashInputBasicStyle]
			sqlite> SELECT * FROM posts WHERE id=2;
		\end{lstlisting}
	\end{itemize}
\end{frame}