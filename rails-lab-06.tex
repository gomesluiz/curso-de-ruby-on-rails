\documentclass[t, 				             
			   final,
			   12pt, 				         
			   xcolor={usenames,dvipsnames}, 
			   table]{beamer}

% pacotes utilizados.
\usepackage[alf]{abntex2cite}
\usepackage{amsmath}
\usepackage[brazil]{babel}
\usepackage{booktabs}
\usepackage{caption}
\usepackage[utf8]{inputenc}
\usepackage{listings}
\usepackage{multicol}
\usepackage{multirow}
\usepackage{todo}


% configuração do tema
\usetheme[pageofpages=de,
          bullet=square,			
          titleline=true,				
          alternativetitlepage=true,			
          titlepagelogo=imagens/logo-puc,	
          watermarkheight=70px,		
          watermarkheightmult=4	
          ]{Torino}

\setbeamertemplate{sections/subsections in toc}[square]
\setbeamertemplate{bibliography item}[default]

\usecolortheme{freewilly}


\definecolor{javared}{rgb}{0.6,0,0} % for strings
\definecolor{javagreen}{rgb}{0.25,0.5,0.35} % comments
\definecolor{javapurple}{rgb}{0.5,0,0.35} % keywords
\definecolor{javadocblue}{rgb}{0.25,0.35,0.75} % javadoc
 
\lstset{}

\lstdefinestyle{BashInputBasicStyle}{
	language=bash,
	basicstyle=\normalsize\ttfamily,
	columns=fullflexible,
	tabsize=2,
	showstringspaces=false,
	frame=single,
	inputencoding=utf8,
	rulecolor=\color{gray}
}

\lstdefinestyle{BashInputStyle}{
  language=bash,
  basicstyle=\normalsize\ttfamily,
  numbers=left,
  numberstyle=\tiny,
  numbersep=2pt,
  frame=tb,
  columns=fullflexible,
  tabsize=2,
  showstringspaces=false,
  commentstyle=\color{gray},
  inputencoding=utf8,
  rulecolor=\color{gray}
}

\lstdefinestyle{RubyInputStyle}{
    language=ruby,
    basicstyle=\scriptsize\ttfamily,
    keywordstyle=\color{javapurple},
    identifierstyle=\color{black},
    commentstyle=\color{javagreen},
	stringstyle=\color{blue},
    showstringspaces=false,
    numbers=left,
    numberstyle=\color{gray}\tiny,
    tabsize=3,
    extendedchars=\true,
    inputencoding=utf8,
%   frame=single, 
    columns=fixed,
    backgroundcolor=\color{red!32!green!33!blue!5}
}    
%  language=ruby,
%  basicstyle=\normalsize\ttfamily,
%  keywordstyle=\color{OrangeRed},
%  identifierstyle=\color{Turquoise},
%  commentstyle=\color{gray},
%  stringstyle=\color{YellowOrange},
%  numbers=left,
%  numberstyle=\tiny,
%  numbersep=2pt,
%  frame=tb,
%  columns=fullflexible,
%  backgroundcolor=\color{white!80},
%  linewidth=0.9\linewidth,
%  tabsize=2,
%  showstringspaces=false
%  inputencoding=utf8


\lstdefinestyle{JavaInputStyle}{
	language=Java,
	basicstyle=\ttfamily,
	keywordstyle=\color{javapurple}\bfseries,
	stringstyle=\color{javared},
	commentstyle=\color{javagreen},
	morecomment=[s][\color{javadocblue}]{/**}{*/},
	numbers=left,
	numberstyle=\tiny\color{black},
	numbersep=10pt,
	tabsize=2,
	showspaces=false,
	showstringspaces=false,
	frame=tb,
	columns=fullflexible,
	backgroundcolor=\color{white!80},
	linewidth=0.9\linewidth,
	inputencoding=utf8
}

\begin{document}
    \author{Luiz Alberto Ferreira Gomes}
    \title{Ruby On Rails: Laboratório 06}
    \subtitle{Laboratório Engenharia de Software}
    \institute{Curso de Ciência da Computação}
    \date{\today}
    
    \include{Capa}
	\include{Agenda}  	

  \section{Bootstrap}
		%-------------------------------------------------------------------------------------- Início
\begin{frame}[fragile, plain, c]{Bootstrap}
    \begin{center}
        \begin{figure}[h]
            \includegraphics[width=0.5\textwidth]{imagens/bootstrap-logo.png}
            \caption{\url{https://getbootstrap.com/}}
        \end{figure}
	\end{center}
\end{frame}

	
	\section{Integrando o Bootstrap}
		%---------------------------------------------------------------------[ Início ]
\begin{frame}[allowframebreaks, fragile,t]{Integrando com Bootstrap}
    \begin{itemize}
      \item Abra o arquivo \verb|Gemfile| na pasta da aplicação
      \item Inclua antes de \verb|gem 'sass-rails', '~> 5'| as seguintes instruções
      \begin{lstlisting}[style=BashInputStyle]
gem 'bootstrap', '~> 4.2.1'
gem 'autoprefixer-rails', '~> 9.6.4'
gem 'jquery-rails', '~> 4.3.5'
      \end{lstlisting}
      \item Execute o comando abaixo para atualizar as dependências
      \begin{lstlisting}[style=BashInputStyle]
~/blog> bundle install 
      \end{lstlisting}
      \item Execute o comando abaixo para renomear o arquivo \verb|application.css| para \verb|application.css.scss|
      \begin{lstlisting}[style=BashInputStyle, basicstyle=\tiny\ttfamily]
~/blog> mv app/assets/stylesheets/application.css  app/assets/stylesheets/application.css.scss
      \end{lstlisting}
      \item Abra o arquivo \verb|application.css.scss| na pasta da aplicação e inclua a seguinte instrução
      \begin{lstlisting}[style=RubyInputStyle, basicstyle=\tiny\ttfamily, caption=app/assets/stylesheets/application.css.scss]
@import bootstrap;
      \end{lstlisting}
  \end{itemize}
\end{frame} 

	\section{Estilizando a Aplicação}
		%---------------------------------------------------------------------[ Início ]
\begin{frame}[allowframebreaks, fragile,t]{Estilizando a Aplicação}
    \begin{itemize}
      \item Modifique as linhas do arquivo \verb|application.html.erb| conforme o código abaixo
      \begin{lstlisting}[style=RubyInputStyle, basicstyle=\tiny\ttfamily, caption=app/views/layouts/application.html.erb, firstline=12, lastline=27]
<!DOCTYPE html>
<html>
  <head>
    <title><%= preenche_titulo yield :titulo %></title>
    <%= csrf_meta_tags %>
    <%= csp_meta_tag %>

    <%= stylesheet_link_tag 'application', media: 'all', 'data-turbolinks-track': 'reload' %>
    <%= javascript_pack_tag 'application', 'data-turbolinks-track': 'reload' %>
  </head>

<body>
  <nav class="navbar navbar-expand-md navbar-dark bg-dark">
    <%= link_to "Blog", "#", id: "logo", class: "navbar-brand" %>
    <ul class="navbar-nav">
      <li class="nav-item">
        <%= link_to "Home", home_index_path, class:"nav-link" %>
      </li>
      <li class="nav-item">
        <%= link_to "Help" , home_help_path , class:"nav-link" %>
      </li>
    </ul>
  </nav>
  <div class="container">
    <%= yield %>
  </div>
</body>
</html>
      \end{lstlisting}
  \end{itemize}
\end{frame} 
		%---------------------------------------------------------------------[ Início ]
\begin{frame}[allowframebreaks, fragile,t]{Estilizando a Visão Home$/$Index}
    \begin{itemize}
      \item Modifique as linhas do arquivo \verb|home/index.html.erb| conforme o código abaixo
      \begin{lstlisting}[style=RubyInputStyle, basicstyle=\tiny\ttfamily, caption=app/views/home/index.html.erb]
<%= provide :titulo, "Home" %>
<div class="center jumbotron">
    <h1>Microblog</h1>
    <p>
    Esta e a pagina principal da aplicação Blog desenvolvida durante 
    o Laboratorio de Engenharia de Software.
    </p>
    <%= link_to "Posts", posts_path, class: "btn btn-lg btn-primary" %>
</div>
      \end{lstlisting}
  \end{itemize}
\end{frame} 
		\include{rails-blog-estilizando-a-visao-posts-index}
		%---------------------------------------------------------------------[ Início ]
\begin{frame}[allowframebreaks, fragile,t]{Estilizando o Partials Posts$/$Form}
    \begin{itemize}
      \item Modifique as linhas do arquivo \verb|posts/_form.html.erb| conforme o código abaixo
      \begin{lstlisting}[style=RubyInputStyle, basicstyle=\tiny\ttfamily, caption=app/views/posts/\_form.html.erb]
<div class="card">
  <%= form_with scope: :post, url: posts_path, 
    local: true do |form| %>
    <div class="card-body">
      <div class="form-group">
        <%= form.label :title %><br>
        <%= form.text_field :title, class: 'form-control' %>
      </div>
      
      <div class="form-group">
        <%= form.label :body %><br>
        <%= form.text_area :body, class: 'form-control', rows: 5 %>
      </div>
      
      <div class="form-group text-right">
        <%= link_to 'Back', posts_path, class: 'btn btn-outline-secondary' %>
        <%= form.submit "Save", class: 'btn btn-primary'%>
      </div>
    </div>
  <% end %>	
</div>
    \end{lstlisting}
  \end{itemize}
\end{frame} 
		\include{rails-blog-estilizando-a-visao-posts-show}

  	\section{Para Saber Mais}
		%%-------------------------------------------------------------------------------------- Início
\begin{frame}[fragile,t]{Para Saber Mais}
  \begin{itemize}
    \item \url{https://www.ruby-lang.org/en/}
    \begin{itemize}
     \item referência oficial da linguagem Ruby onde a toda a sua documentação está disponível
	para ser consultada.
    \end{itemize}

    \item \url{http://rubyonrails.org/}
    \begin{itemize}
     \item referência oficial do framework Rails onde a toda a sua documentação está disponível
	para ser consultada.
    \end{itemize}
    
    \item \url{http://www.codecademy.com/pt/tracks/ruby}
    \begin{itemize}
     \item curso iterativo em portugês sobre a linguagem Ruby.
    \end{itemize}

	\item \url{https://gorails.com/setup/ubuntu/16.04}
	\begin{itemize}
		\item guia para instalação do Ruby on Rails no Ubuntu e no Mac OSX.
	\end{itemize}
  \end{itemize}
  
  
\end{frame}
\end{document}
