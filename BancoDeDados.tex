%\section{Banco de Dados Relacionais}
\begin{frame}[t, allowframebreaks, fragile]{Banco de Dados Relacionais}
	\begin{itemize}
		\item Um aspecto importante da programação web é a habilidade de coletar, armazenar e recuperar
		diferentes formas de dados
		\begin{itemize}
			\item uma das formas mais populares são os \alert{bancos de dados relacionais}
		\end{itemize} 
% 		\item maioria banco de dados relacionais acessados através da Structured Query Language (SQL)
		\item Um banco de dados relacional é baseado entidades, denominadas \alert{tabelas}, no
		relacionamento, \alert{associações}, entre elas
		\item O contêiner fundamental em um banco de dados relacional é denominado de \alert{database} ou \alert{schema}
		\begin{itemize}
			\item podem incluir estruturas de dados, os dados propriamente ditos e permissões de acesso
		\end{itemize}  
		\framebreak
		\item Os dados são armazenados em \alert{tabelas} e as tabelas são divididas em \alert{linhas} e \alert{colunas}.
		Por exemplo:
		\begin{table}[tp] 
			\scriptsize 
			\caption{comment}
			\setlength{\tabcolsep}{8pt}
			\setlength{\extrarowheight}{2pt}   			
			\begin{tabular}{|l|l|l|} 
				\hline
				\textbf{id} & \textbf{post\_id} & \textbf{body}\\
				\hline
				10 & 1 & Ruby realmente... \\
				\hline
				11 & 2 & Rails facilita... \\
				\hline
				13 & 2 & Concordo, ... \\
				\hline
			\end{tabular}
		\end{table}
		\pagebreak
		\item Relacionamentos são estabelecidos entre tabelas para que a consistência dos dados
		seja mantida em qualquer situação e podem ser:
		\begin{itemize}
			\item 1:1
			\item 1:N
			\item N:M
			\begin{columns}[t]
				\column{0.4\textwidth}
				\begin{table}[tp] 
					\scriptsize 
					\caption{comment}
					\setlength{\tabcolsep}{8pt}
					\setlength{\extrarowheight}{2pt}   			
					\begin{tabular}{|l|l|l|} 
						\hline
						\textbf{id} & \textbf{post\_id} & \textbf{body}\\
						\hline
						10 & 1 & Ruby realmente... \\
						\hline
						11 & 2 & Rails facilita... \\
						\hline
						13 & 2 & Concordo, ... \\
						\hline
					\end{tabular}
				\end{table}
				\column{0.6\textwidth}
				\begin{table}[tp] 
					\caption{post}
					\scriptsize 
					\setlength{\tabcolsep}{8pt}
					\setlength{\extrarowheight}{1pt}   			
					\begin{tabular}{|l|l|l|} 
						\hline
						\textbf{id} & \textbf{title} & \textbf{body} \\
						\hline
						\textbf{1} & A Linguagem Ruby & Ruby é legal. \\
						\hline
						\textbf{2} & O Framework Rais & O Rais facilita...\\
						\hline
					\end{tabular}
				\end{table}   		
			\end{columns}
		\end{itemize}  
	\end{itemize}
\end{frame}