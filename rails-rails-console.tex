\begin{frame}[allowframebreaks, t, fragile]{Rails Console}
	\begin{itemize}
		\item Inicialize \alert{na pasta da aplicação} a console do Rails (não a do banco de dados):
		\begin{lstlisting}[style=BashInputBasicStyle]
			$ rails c
		\end{lstlisting}
		
		\item Exiba os atributos da classe \verb|Post|:
		\begin{lstlisting}[style=BashInputBasicStyle]
			irb(main):004:0> Post.column_names
		\end{lstlisting}
		
		\item Crie um novo \verb|post| e salve no banco de dados:
		\begin{lstlisting}[style=BashInputBasicStyle]
			irb(main):005:0> p1 = Post.new
			irb(main):006:0> p1.title="Rails is Cool!"
			irb(main):007:0> p1.body="Rails is really Cool..."
			irb(main):008:0> p1.save
		\end{lstlisting}
		
		\item Exiba todos os \verb|posts|:
		\begin{lstlisting}[style=BashInputBasicStyle]
			irb(main):007:0> Post.all
		\end{lstlisting}
		
		\item Exiba todos os \verb|posts| ordenados pelo título (title):
		\begin{lstlisting}[style=BashInputBasicStyle]
			irb(main):007:0> Post.all.order(title: :asc)
		\end{lstlisting}
		
		\item Exiba um \verb|post|:
		\begin{lstlisting}[style=BashInputBasicStyle]
			irb(main):007:0> Post.first
		\end{lstlisting}
		
		\item Exiba o \verb|post| cujo \verb|id| é 2:
		\begin{lstlisting}[style=BashInputBasicStyle]
			irb(main):007:0> Post.find_by(id: 2)
		\end{lstlisting}
		
		\item Atualize o título do primeiro \verb|post|:
		\begin{lstlisting}[style=BashInputBasicStyle]
			irb(main):007:0> p1=Post.first
			irb(main):008:0> p1.update(title: "Rails rules!")
		\end{lstlisting}
		
		\item Remova do primeiro \verb|post|:
		\begin{lstlisting}[style=BashInputBasicStyle]
			irb(main):007:0> p1=Post.first
			irb(main):008:0> p1.destroy
		\end{lstlisting}
		
	\end{itemize}
\end{frame}