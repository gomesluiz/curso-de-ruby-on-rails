\begin{frame}[allowframebreaks, t, fragile]{I3 - Hora de Colocar a Mão na Massa}
	\begin{itemize}
		\item Modifique o arquivo \verb|app/models/post.rb| para associar 
		o post aos seus comentário:
		\begin{lstlisting}[style=RubyInputStyle]
			class Post < ApplicationRecord
			  validates_presence_of :title, :body
			  has_many :comments
			end
		\end{lstlisting}
		
		\item Modifique o arquivo \verb|app/models/comment.rb| para associar 
		o comentário ao seu post:
		\begin{lstlisting}[style=RubyInputStyle]
			class Comment < ApplicationRecord
			  validates_presence_of :body
			  belongs_to :post
			end
		\end{lstlisting}
		
%		\item Crie um novo \verb|post| e salve no banco de dados (\alert{Reinicie a console do Rails}):
%		\begin{lstlisting}[style=BashInputBasicStyle]
%			irb(main):005:0> p1 = Post.new
%			irb(main):006:0> p1.title="Associacao"
%			irb(main):007:0> p1.body="Eu tenho comentarios!"
%			irb(main):008:0> p1.save
%		\end{lstlisting}
		
%		\item Crie um novo \verb|comment| e o vincule a um \verb|post|:
%		\begin{lstlisting}[style=BashInputBasicStyle]
%			irb(main):005:0> c1 = Comment.new
%			irb(main):006:0> c1.body="Eu sou de um post!"
%			irb(main):007:0> c1.post = p1
%			irb(main):008:0> c1.save
%		\end{lstlisting}
		
%		\item Consulte os comentários do \verb|post| p1:
%		\begin{lstlisting}[style=BashInputBasicStyle]
%			irb(main):005:0> p1.comments.all
%		\end{lstlisting}
		
%		\item Consulte os comentários 2 do \verb|post| p1:
%		\begin{lstlisting}[style=BashInputBasicStyle]
%			irb(main):005:0> p1.comments.where(id: 2)
%		\end{lstlisting}
		
%		\item Consulte o \verb|post| do comentário c1:
%		\begin{lstlisting}[style=BashInputBasicStyle]
%			irb(main):005:0> c1.post
%		\end{lstlisting}
					
	\end{itemize}
\end{frame}