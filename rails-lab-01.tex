\documentclass[t, 				             
			   final,
			   12pt, 				         
			   xcolor={usenames,dvipsnames}, 
			   table]{beamer}

% pacotes utilizados.
\usepackage[alf]{abntex2cite}
\usepackage{amsmath}
\usepackage[brazil]{babel}
\usepackage{booktabs}
\usepackage{caption}
\usepackage[utf8]{inputenc}
\usepackage{listings}
\usepackage{multicol}
\usepackage{multirow}
\usepackage{todo}


% configuração do tema
\usetheme[pageofpages=de,
          bullet=square,			
          titleline=true,				
          alternativetitlepage=true,			
          titlepagelogo=imagens/logo-puc,	
          watermarkheight=70px,		
          watermarkheightmult=4	
          ]{Torino}

\setbeamertemplate{sections/subsections in toc}[square]
\setbeamertemplate{bibliography item}[default]

\usecolortheme{freewilly}


\definecolor{javared}{rgb}{0.6,0,0} % for strings
\definecolor{javagreen}{rgb}{0.25,0.5,0.35} % comments
\definecolor{javapurple}{rgb}{0.5,0,0.35} % keywords
\definecolor{javadocblue}{rgb}{0.25,0.35,0.75} % javadoc
 
\lstset{}

\lstdefinestyle{BashInputBasicStyle}{
	language=bash,
	basicstyle=\normalsize\ttfamily,
	columns=fullflexible,
	tabsize=2,
	showstringspaces=false,
	frame=single,
	inputencoding=utf8,
	rulecolor=\color{gray}
}

\lstdefinestyle{BashInputStyle}{
  language=bash,
  basicstyle=\normalsize\ttfamily,
  numbers=left,
  numberstyle=\tiny,
  numbersep=2pt,
  frame=tb,
  columns=fullflexible,
  tabsize=2,
  showstringspaces=false,
  commentstyle=\color{gray},
  inputencoding=utf8,
  rulecolor=\color{gray}
}

\lstdefinestyle{RubyInputStyle}{
    language=ruby,
    basicstyle=\scriptsize\ttfamily,
    keywordstyle=\color{javapurple},
    identifierstyle=\color{black},
    commentstyle=\color{javagreen},
	stringstyle=\color{blue},
    showstringspaces=false,
    numbers=left,
    numberstyle=\color{gray}\tiny,
    tabsize=3,
    extendedchars=\true,
    inputencoding=utf8,
%   frame=single, 
    columns=fixed,
    backgroundcolor=\color{red!32!green!33!blue!5}
}    
%  language=ruby,
%  basicstyle=\normalsize\ttfamily,
%  keywordstyle=\color{OrangeRed},
%  identifierstyle=\color{Turquoise},
%  commentstyle=\color{gray},
%  stringstyle=\color{YellowOrange},
%  numbers=left,
%  numberstyle=\tiny,
%  numbersep=2pt,
%  frame=tb,
%  columns=fullflexible,
%  backgroundcolor=\color{white!80},
%  linewidth=0.9\linewidth,
%  tabsize=2,
%  showstringspaces=false
%  inputencoding=utf8


\lstdefinestyle{JavaInputStyle}{
	language=Java,
	basicstyle=\ttfamily,
	keywordstyle=\color{javapurple}\bfseries,
	stringstyle=\color{javared},
	commentstyle=\color{javagreen},
	morecomment=[s][\color{javadocblue}]{/**}{*/},
	numbers=left,
	numberstyle=\tiny\color{black},
	numbersep=10pt,
	tabsize=2,
	showspaces=false,
	showstringspaces=false,
	frame=tb,
	columns=fullflexible,
	backgroundcolor=\color{white!80},
	linewidth=0.9\linewidth,
	inputencoding=utf8
}

\begin{document}
    \author{Luiz Alberto Ferreira Gomes}
    \title{Ruby On Rails: Laboratório 01}
    \subtitle{Laboratório de Engenharia de Software}
    \institute{Curso de Ciência da Computação}
    \date{\today}

	\include{Capa}
	\include{Agenda}  	
    
    \section{Ruby on Rails}
	%-------------------------------------------------------------------------------------- Início
\begin{frame}[fragile, plain, c]{Rails}
	\begin{center}
		\large Rails é um \alert{framework} para construção de \alert{aplicações web} baseado na \alert{linguagem Ruby}.
	\end{center}
\end{frame}
%-------------------------------------------------------------------------------------- Início
\begin{frame}[fragile, t]{Rails}
	\begin{itemize}
		\item Uma \alert{gem} (pacote) escrita em Ruby
		\item 100\% \alert{open-source}, MIT License
		\item Fornece \alert{geradores} de código e scripts de \alert{automação} de testes
		\item Ferramentas adicionais são fornecidas no ecossistema Rails:
	\end{itemize}
\end{frame}

%-------------------------------------------------------------------------------------- Início
\begin{frame}[fragile, t]{Ferramentas Adicionais}
	\begin{itemize}
		\item \alert{Rake} - utilitário similar ao \textbf{make do Unix} para criar e migrar bancos de dados, limpar sessões de uma Web app
		\item \alert{Puma} - servidor web de desenvolvimento para execução de aplicações Rails
		\item \alert{SQLite} - um servidor de banco de dados simples pré-instalado como o Rails
		\item \alert{Rack Middleware} - interface padronizado para interação entre um servidor web e uma Web App
	\end{itemize}
\end{frame}	
	%%\section{Histórico de Evolução}
%-------------------------------------------------------------------------------------- Início
\begin{frame}[allowframebreaks,fragile,t]{Histórico do Rails}
  \begin{itemize}
  	\item Rails é um \textit{framework} para construção de aplicações web
    \item David Heinemeier Hanson \alert{derivou} o Rails a partir do BaseCamp --
      uma ferramenta de gestão de projetos da empresa 37Signals.
    \begin{itemize}
	\item a primeira versão de código aberto (em inglês: \textit{open source})foi liberada em 
	  julho de 2004.
	\item mas direitos para que outros desenvolvedores \alert{colaborassem} com o projeto foram liberadosw
	  em fevereiro de 2005.
    \end{itemize}
    \item Em agosto de 2006, o Ruby on Rails atingiu um \alert{marco importante} quando a Apple dicidiu
      distribuído juntamente com a versão do seu sistema operacional Mac OS X v10.5 "Leopard"
    \begin{itemize}
     \item nesse mesmo no o Rails começou a ganhar muita atenção da comunidade de desenvolvimento web.
    \end{itemize}
    \item Rails é utilizado por diversas companhias, como por exemplo:
    \begin{itemize}
     \item Airbnb, BaseCamp, Disney, GitHub, Hulu, Kickstarter, Shopify e Twitter.
    \end{itemize}

  \end{itemize}

  \begin{figure}[h]
    \includegraphics[scale=0.3]{imagens/rails-history.png}  
  \end{figure}
\end{frame}

	\begin{frame}[allowframebreaks,fragile,t]{Filosofia do Rails}
  \begin{itemize}
    \item Ruby on Rails é 100\% open-source, disponível por meio da MIT License:
      \url{http://opensource.org/licenses/mit-license.php}.  
    \item \alert{Convenção} acima da Configuração (em inglês: \textit{Convention over Configuration} (CoC))
    \begin{itemize}
	\item se nomeação segue certas convenções, não há necessidade de arquivos de configuração.
	\begin{exampleblock}{Exemplo:}
	  \begin{verbatim}
      FilmesController#show -> filmes_controler.rb 
      FilmesController#show -> views/filmes/show.html.er
    \end{verbatim}
	\end{exampleblock}
    \end{itemize}
    \item \alert{"Don't Repeat Yourself"} (DRY) sugere que escrever que o mesmo código várias 
      vezes é uma coisa ruim
    
    \item O \textit{Representational State Transfer} (REST) é o melhor padrão para desenvolvimento de aplicações web
    \begin{itemize}
     \item organiza a sua aplicação em torno de \alert{recursos} e \alert{padrões} HTTP (verbs)
    \end{itemize}
  \end{itemize}   
\end{frame}

	%%\section{MVC em Ação no Rails}
%%-------------------------------------------------------------------------------------- Início
\begin{frame}[t, fragile]{Model-View-Controller}
	\begin{itemize}
		\item O framework Rails é contruído em cima do Design Pattern Model View Controller(MVC):
	\end{itemize}
	\begin{figure}[h!]
		\centering
		\includegraphics[width=0.70\textwidth]{imagens/mvc.jpg}
	\end{figure}
\end{frame}
 

    
    \section{Metodologia de Trabalho}
    \begin{frame}[fragile,t]{Metodologia de Trabalho}
	\begin{figure}[h!]
		\centering
		\includegraphics[width=.9\textwidth]{imagens/metodologia-de-trabalho.jpg}
	\end{figure}
\end{frame}

\begin{frame}[allowframebreaks, fragile,t]{Passos Iniciais do Blog App}
	\begin{enumerate}
	    \item Inicie uma janela de terminal e digite no prompt:
	     \begin{lstlisting}[style=BashInputBasicStyle]
		     $ cd 
		     $ rails new blog
	     \end{lstlisting}
%	    \item Utilize o gerador \verb|scaffold| para criar os 
%	    componentes MVC para as postagens e os comentários
%	     \begin{lstlisting}[style=BashInputBasicStyle]
%					$ rails generate scaffold post \ 
%			     title:string body:text
%					$ rails generate scaffold comment post_id:integer \
%					 body:text
%	     \end{lstlisting}
%	    \item Gere as tabelas \verb|post| e \verb|comment| no banco de dados
%	    \begin{lstlisting}[style=BashInputBasicStyle]
%		    $ rake db:migrate
%	    \end{lstlisting}
%	    
%	    \item Visualize todas as URLs reconhecidas pela sua aplicação digitando:
%	    \begin{lstlisting}[style=BashInputBasicStyle]
%	    $ rake routes
%	    \end{lstlisting}
	     \item Inicie o servidor web embutido:
	     \begin{lstlisting}[style=BashInputBasicStyle]
	     $ rails s
	     \end{lstlisting}
	     
	     \item Abra uma janela do navegador e digite:
	     \begin{lstlisting}[style=BashInputBasicStyle]
	     $ http://localhost:3000
	     \end{lstlisting}
	\end{enumerate}
  
\end{frame}


    \section{Esqueleto da Aplicação}
    
\begin{frame}[allowframebreaks,fragile,t]{Estrutura de uma Aplicação Rails}
  \begin{figure}[h]
    \includegraphics[scale=0.35]{imagens/rails-esqueleto-01.png}  
  \end{figure}
  \begin{figure}[h]
    \includegraphics[scale=0.35]{imagens/rails-esqueleto-02.png}  
  \end{figure}
  \begin{figure}[h]
    \includegraphics[scale=0.35]{imagens/rails-esqueleto-03.png}  
  \end{figure}
\end{frame}
   
    \section{Pŕatica}
    
\begin{frame}[fragile,t]{Prática de Laboratório (Para Casa)}
  \begin{itemize}
      \item Instale no seu computador ou configure o ambiente Plaza Cloud para as 
      próximas práticas de laboratório.
  \end{itemize}
\end{frame}

    \section{Para Saber Mais}
    %%-------------------------------------------------------------------------------------- Início
\begin{frame}[fragile,t]{Para Saber Mais}
  \begin{itemize}
    \item \url{https://www.ruby-lang.org/en/}
    \begin{itemize}
     \item referência oficial da linguagem Ruby onde a toda a sua documentação está disponível
	para ser consultada.
    \end{itemize}

    \item \url{http://rubyonrails.org/}
    \begin{itemize}
     \item referência oficial do framework Rails onde a toda a sua documentação está disponível
	para ser consultada.
    \end{itemize}
    
    \item \url{http://www.codecademy.com/pt/tracks/ruby}
    \begin{itemize}
     \item curso iterativo em portugês sobre a linguagem Ruby.
    \end{itemize}

	\item \url{https://gorails.com/setup/ubuntu/16.04}
	\begin{itemize}
		\item guia para instalação do Ruby on Rails no Ubuntu e no Mac OSX.
	\end{itemize}
  \end{itemize}
  
  
\end{frame}
\end{document}
