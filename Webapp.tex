\section{Aplicação Web}
%%-------------------------------------------------------------------------------------- Início

%%-------------------------------------------------------------------------------------- Início
\begin{frame}[allowframebreaks, fragile,t]{Aplicação Web}
  \begin{exampleblock}{Definição(Aplicação Web)}
    Uma \alert{aplicação web} é aquela que acessada pelos usuários por meio de uma \alert{rede de computadores}, utiliza
    um \alert{navegador} (em inglês: \textit{browser}); e consiste de uma coleção de \alert{\textit{scripts}} no cliente e 
    no servidor, páginas \alert{HTML} e outros recursos que podem estar espalhados por vários servidores. Ele é 
    acessada pelos usuários via \alert{um endereço} que faz referência a um servidor web (por exemplo: www.inf.pucpcaldadas.br).
  \end{exampleblock}
  
  \begin{itemize}
    \item Exemplos: webmail, lojas virtuais, homebanking, wikis, blogs e etc.
  \end{itemize}
  
\framebreak

  \begin{itemize}
    \item Há um pouco mais do que isso:
    \begin{itemize}
      \item Rede de Computadores:
      \begin{itemize}
        \item a \alert{Internet}, um sistema global de redes de computadores interconectadas.
        \item utiliza o conjunto de protocolos TCP/IP.
      \end{itemize}
      \item Web (World Wide Web):
      \begin{itemize}
	\item um sistema de documentos (em inglês: \textit{web pages}) \alert{vinculados} que são acessados através 
	  da Internet via protocolo HTTP.
	\item Web pages contêm documentos \alert{hypermedia}: textos, gráficos, imagens, vídeos e outros recursos multimídia, juntamente com \textit{hiperlinks} para outras páginas
	\item \alert{Hiperlinks} formam a \alert{estrutura básica} da Web.
	\item A estrutura da Web é a que a torna \alert{útil} e de \alert{valor}.
      \end{itemize}
    \end{itemize}
    \item \underline{Vantagens}:
    \begin{itemize}
      \item \alert{Conveniência} pela utilização um web browser como cliente. 
      \item \alert{Compatibilidade} inerente entre plataformas.
      \item Habilidade de \alert{atualizar} e \alert{manter} as aplicações web sem instalação e distribuição de software
        em vários clientes em potencial.
      \item \alert{Redução} dos custos de TI.
    \end{itemize}
    \item \alert{Desvantagens}:
    \begin{itemize}
      \item Interfaces com usuário ainda \alert{não são tão boas} quanto as das aplicações tradicionais.
      \item Maior risco de \alert{comprometimento} da \alert{privacidade} e \alert{segurança dos dados}.
      \item Mais \alert{difícil} de \alert{desenvolver} e \alert{depurar} do que uma aplicação tradicional, pois 
        existem mais partes a se considerar.
    \end{itemize}
  \end{itemize}
\end{frame}
%%-------------------------------------------------------------------------------------- Início