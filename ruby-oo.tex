
\begin{frame}[fragile,t]{OO}
  \begin{itemize}
    \item OO possibilita identificar "coisas" que serão tratadas pelo programa
    \item \alert{Classes} são descrições dessas \alert{"coisas"} e container de métodos
    \item Objetos são \alert{instâncias} dessas classes
    \item Objetos contêm \alert{variáveis de instância} (estado)
  \end{itemize}
\end{frame}

\begin{frame}[fragile,t]{Variáveis de Instância}  
 \begin{itemize}
    \item Iniciam com: \alert{@}
    \begin{itemize}
      \item exemplo: @nome
    \end{itemize}
    \item Não há necessidade de declará-las 
    \item Disponível para todas as instâncias dos métodos da classe
  \end{itemize}
\end{frame}

\begin{frame}[allowframebreaks, fragile,t]{Criação de Objetos}    
  \begin{itemize}
    \item Classes são fábricas
    \begin{itemize}
      \item \alert{new} cria uma instância da classe e invoca o método \alert{initialize}
      \item O estado do objeto deve ser inicializado no método \alert{initialize} (construtor)
    \end{itemize}
  \end{itemize}  
  
  \lstinputlisting[style=RubyInputStyle, caption=classes.rb]{codigos/ruby/10-classes/classes.rb}
\end{frame}

\begin{frame}[allowframebreaks, fragile,t]{Acesso a Variáveis de Instância}    
  \begin{itemize}
    \item Variáveis de instância são \alert{privadas}
    \item Métodos são públicos por padrão
    \item Getters/setters para acessar variáveis de instância são necessários
  \end{itemize}  
  \lstinputlisting[style=RubyInputStyle, caption=instance\_vars.rb]{codigos/ruby/10-classes/instance_vars.rb}
  
  \begin{itemize}
    \item Muitas vezes as lógicas dos getters/setters são muito simples
    \item Existe uma maneira mais fácil de definir esses métodos em Ruby
    \begin{itemize}
      \item \alert{attr\_accessor} - getter e setter
      \item \alert{attr\_reader} - somente getter
      \item \alert{attr\_writer} - somente setter
    \end{itemize}
  \end{itemize}  
  \lstinputlisting[style=RubyInputStyle, caption=attr\_accessor.rb]{codigos/ruby/10-classes/attr_accessor.rb}

  \begin{itemize}
    \item \alert{Dois problemas} com o exemplo acima:
    \begin{itemize}
      \item Pessoa se encontra em um estado não inicializado na criação
      \item Algumas vezes é necessário controlar, por exemplo, a idade atribuída
    \end{itemize}
    \item \alert{Solução}: use o construtor de forma mais inteligente utilizando o comando \alert{self}
  \end{itemize}  
  \lstinputlisting[style=RubyInputStyle, caption=self.rb]{codigos/ruby/10-classes/self.rb}
 
\end{frame}

\begin{frame}[allowframebreaks, fragile,t]{Métodos e Variáveis de Classe}    
  \begin{itemize}
    \item Use \alert{self} para definir métodos de classe
    \item Variáveis de classe começam com \alert{@@}
  \end{itemize}  
  \lstinputlisting[style=RubyInputStyle, caption=class\_methods\_and\_variables.rb]{codigos/ruby/11-more-classes/class_methods_and_variables.rb}
 
\end{frame}

\begin{frame}[allowframebreaks, fragile,t]{Herança de Classes}    
%  \begin{itemize}
%    \item \alert{Existem} três maneiras para definir métodos de classe
%    \item Variáveis de classe começam com \alert{@@}
%  \end{itemize}  
  \lstinputlisting[style=RubyInputStyle, caption=inheritance.rb]{codigos/ruby/11-more-classes/inheritance.rb}
 
 %TODO: controle de acesso privado, publico e protegido.
\end{frame}

%-------------------------------------------------------------------------------------- Início
\begin{frame}[allowframebreaks, fragile, t]{Exercícios}
Elabore na linguagem JavRuby os códigos para os seguintes requisitos:
    \begin{enumerate}
          \item Escreva a classe \textbf{Sobremesa} com \textit{getters} e \textit{setters} para os atributos \texttt{nome} e \texttt{calorias}. O construtor dessa classe deverá receber como parâmetros \texttt{nome} e \texttt{calorias}.      
          \item Defina as operações de instância \texttt{ehSaudavel}, que retorna \texttt{true} se e somente se a sobremesa tem menos de 200 calorias, e \texttt{ehDeliciosa}, que retorna \texttt{true} para todas as sobremesas.
          \item Crie a classe \textbf{GeleiaEmCompota} que herdará da classe \textbf{Sobremesa}. O seu construtor deverá aceitar um único argumento denominado \texttt{sabor};  a sua quantidade padrão de \texttt{calorias} é 5 e seu \texttt{nome} deverá ser precedido de "Geléia em Compota de ", por exemplo, "Geléia em Compota de Morango".
          \item Inclua um \textit{getter} and \textit{setter} para o atributo \texttt{sabor}.   
          \item Modifique a operação \texttt{ehDeliciosa} para retornar \textbf{false} se o sabor é \texttt{alcaçuz} e true para todos os outros sabores. O comportamento dessa operação para sobremesas que não são geléias em compotas não devem ser alterados.
       \end{enumerate}  
    
\end{frame}


\begin{frame}[fragile,t]{Recapitulando}
  \begin{itemize}
    \item Objetos são criados com \alert{new}
    \item Utilize o \alert{attr\_} para criar getters/setters
    \item Não se esqueça do \alert{self} quando necessário
    \item Variáveis de classe são definidas com \alert{@@}
  \end{itemize}
\end{frame}



