%%-------------------------------------------------------------------------------------- Início
\begin{frame}[allowframebreaks, t, fragile]{Form Helpers}
	\begin{itemize}
		\item \alert{form_for} gere a tag form para o objeto passado como parâmetro
		\item Rails utiliza a método \alert{POST} por padrão
		\item Isto faz sentido:
		\begin{itemize}
			\item uma password não é passada como parâmetro na URL
			\item qualquer modificação deverá ser feita via POST e não GET
		\end{itemize}
	\end{itemize}	
\end{frame}
%%-------------------------------------------------------------------------------------- Início
\begin{frame}[allowframebreaks, t, fragile]{f.label}
	\begin{itemize}
		\item Gera a tag HTML label 
		\item A descrição pode ser personalizada passando um segundo parâmetro
		\begin{lstlisting}[style=RubyInputStyle, caption=posts_controller.rb]
			.. f.label
			
		\end{lstlisting}	
	\end{itemize}	
\end{frame}
%%-------------------------------------------------------------------------------------- Início
\begin{frame}[allowframebreaks, t, fragile]{f.text_field}
	\begin{itemize}
		\item Gera o campo input type="text" 
		\item Utilize \alert{:placeholder} para mostrar um valor dentro do campo
		\begin{lstlisting}[style=RubyInputStyle, caption=posts_controller.rb]
			.. f.text_field
			
		\end{lstlisting}	
	\end{itemize}	
\end{frame}
%%-------------------------------------------------------------------------------------- Início
\begin{frame}[allowframebreaks, t, fragile]{f.text_area}
	\begin{itemize}
		\item Similar ao \alert{f.text_field}, mas gera um text area de tamanho (40 cols x 20 rows) 
		\item O tamanho pode ser modificado através do atriburo size:
		\begin{lstlisting}[style=RubyInputStyle, caption=posts_controller.rb]
			.. f.text_area
			
		\end{lstlisting}	
	\end{itemize}	
\end{frame}
%%-------------------------------------------------------------------------------------- Início
\begin{frame}[allowframebreaks, t, fragile]{Outros Form Helpers}
	\begin{itemize}
		\item date_select 
		\item search_field
		\item telephone_field
		\item utl_field
		\item email_field
		\item numer_field
		\item range_field	
	\end{itemize}	
\end{frame}
%%-------------------------------------------------------------------------------------- Início
\begin{frame}[allowframebreaks, t, fragile]{f.submit}
	\begin{itemize}
		\item Renderiza o botão \alert{submit} 
		\item Aceita o nome do botão submit como primeiro argumento
		\item Se o nome não for fornecido - gera um baseado no modelo e na ação. Por exemplo:
			"Create Post" ou "Update Post" 
		\item Mais form helpers: \url{http://guides.rubyonrails.org/form_helpers.html}	
	\end{itemize}	
\end{frame}