%%-------------------------------------------------------------------------------------- Início
\begin{frame}[allowframebreaks, t, fragile]{Form Helpers}
	\begin{itemize}
		\item \alert{form\_for} gere a tag form para o objeto passado como parâmetro
		\item Rails utiliza a método \alert{POST} por padrão
		\item Isto faz sentido:
		\begin{itemize}
			\item uma password não é passada como parâmetro na URL
			\item qualquer modificação deverá ser feita via POST e não GET
		\end{itemize}
	\end{itemize}	
	\begin{lstlisting}[style=RubyInputStyle, caption=views/posts/\_form.html.erb]
		<%= form_for(@post) do |f| %>
		...
		<% end %>		
	\end{lstlisting}
\end{frame}

%%-------------------------------------------------------------------------------------- Início
\begin{frame}[t, fragile]{f.label}
	\begin{itemize}
		\item Gera a tag HTML \alert{label} 
		\item A descrição pode ser \alert{personalizada} passando um segundo parâmetro
		\begin{lstlisting}[style=RubyInputStyle]
<div class="field">
  <%= f.label :title, "Titulo" %><br>
  <%= f.text_field :title %>
</div>
		\end{lstlisting}	
	\end{itemize}	
\end{frame}
%%-------------------------------------------------------------------------------------- Início
\begin{frame}[t, fragile]{f.text\_field}
	\begin{itemize}
		\item Gera o campo \alert{input type="text"} 
		\item Utilize \alert{:placeholder} para mostrar um valor dentro do campo
		\begin{lstlisting}[style=RubyInputStyle]
<div class="field">
  <%= f.label :title, "Titulo" %><br>
  <%= f.text_field :title, placeholder: "Escreva o titulo aqui." %>
</div>		
		\end{lstlisting}	
	\end{itemize}	
\end{frame}
%%-------------------------------------------------------------------------------------- Início
\begin{frame}[t, fragile]{f.text\_area}
	\begin{itemize}
		\item Similar ao \alert{f.text\_field}, mas gera um text area de tamanho (40 cols x 20 rows) 
		\item O tamanho pode ser modificado através do atriburo size:
		\begin{lstlisting}[style=RubyInputStyle]
<div class="field">
  <%= f.label :body, "Conteudo" %><br>
  <%= f.text_area :body, size: "10x3" %>
</div>			
		\end{lstlisting}	
	\end{itemize}	
\end{frame}
%%-------------------------------------------------------------------------------------- Início
\begin{frame}[t, fragile]{Outros Form Helpers}
	\begin{itemize}
		\item date\_select 
		\item search\_field
		\item telephone\_field
		\item url\_field
		\item email\_field
		\item number\_field
		\item range\_field	
	\end{itemize}	
\end{frame}
%%-------------------------------------------------------------------------------------- Início
\begin{frame}[t, fragile]{f.submit}
	\begin{itemize}
		\item Renderiza o botão \alert{submit} 
		\item Aceita o \alert{nome} do botão submit como primeiro argumento
		\item Se o nome não for fornecido - gera um baseado no modelo e na ação. Por exemplo:
			"Create Post" ou "Update Post" 
			\begin{lstlisting}[style=RubyInputStyle]
<div class="actions">
  <%= f.submit "Postar"%>
</div>			
			\end{lstlisting}	
		\item Mais form helpers: \url{http://guides.rubyonrails.org/form_helpers.html}	
	\end{itemize}	
\end{frame}