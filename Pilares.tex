%%\subsection{Filosofia do Rails}
%%-------------------------------------------------------------------------------------- Início
\begin{frame}[allowframebreaks,fragile,t]{Filosofia do Rails}
  \begin{itemize}
    \item Ruby on Rails é 100\% open-source, disponível por meio da MIT License:
      \url{http://opensource.org/licenses/mit-license.php}.  
 
    \item \alert{Convenção} acima da Configuração (em inglês: \textit{Convention over Configuration} (CoC))
    \begin{itemize}
	\item se nomeação segue certas convenções, não há necessidade de arquivos de configuração.
	\begin{exampleblock}{Exemplo:}
	  \begin{verbatim}
	    FilmesController#show -> filmes_controler.rb 
	    FilmesController#show -> views/filmes/show.html.erb
	  \end{verbatim}
	\end{exampleblock}
    \end{itemize}
  
    \item \alert{"Don't Repeat Yourself"} (DRY) sugere que escrever que o mesmo código várias 
      vezes é uma coisa ruim
    
    \item O \textit{Representational State Transfer} (REST) é o melhor padrão para desenvolvimento de aplicações web
    \begin{itemize}
     \item organiza a sua aplicação em torno de \alert{recursos} e \alert{padrões} HTTP (verbs)
    \end{itemize}

  \end{itemize}   
\end{frame}
