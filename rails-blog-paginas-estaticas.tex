%---------------------------------------------------------------------[ Início ]
\begin{frame}[allowframebreaks, fragile,t]{Controladores e Ações}
    \begin{itemize}
      \item Um controlador Rails pode ser gerado utilizando o comando \alert{rails 
	generate controller} ou \alert{rails g controller}.Além dele, esse comando permite 
	a geração de algumas ações.
    \end{itemize}
      \begin{lstlisting}[style=BashInputStyle]
 ~/blog$ rails generate controller Home index help
 route  get 'home/help'
 route  get 'home/index'
 invoke  erb
 create    app/views/home
 create    app/views/home/index.html.erb
 create    app/views/home/help.html.erb
 ...
      \end{lstlisting}
\end{frame}

%---------------------------------------------------------------------[ Início ]
\begin{frame}[allowframebreaks, fragile,t]{Arquivo de Roteamento: routes.rb}
    \begin{itemize}
      \item O conteúdo atualizado do arquivo \alert{routes.rb} deverá conter as notas rotas 
	para as ações recém-criadas.
	  \begin{lstlisting}[style=RubyInputStyle, caption=config/routes.rb]
	    Rails.application.routes.draw do
	      get 'home/index'
	      get 'home/help'
	      root 'home#index'
	      resources :posts 
	    end 
	  \end{lstlisting}
    
      \item \alert{GET} é uma das operações do HTTP mais comuns, ela é utilizada para a leitura
      de dados na web.
      \begin{itemize}
	\item o navegador envia uma requisição \alert{GET} toda vez de se deseja visitar um endereço, com por exemplo \url{http://www.inf.pucpcaldas.br}. 
       
      \end{itemize}

      \item As páginas vinculadas às ações \alert{index} e \alert{help} podem ser 
	visitadas, primeiro, iniciando o servidor Rails com os comandos \alert{rails server} ou, 
	simplesmente, \alert{rails s}.
	
	\begin{lstlisting}[style=BashInputStyle]
	  $ cd ~/workspace/blog
	  $ rails server
	\end{lstlisting}
    
      \item E depois, digitando as páginas os endereços \alert{\url{http://localhost:3000/home/index}} e, depois, \alert{\url{http://localhost:3000/home/help}}.
      \item Dica: Introdução às Rotas do Rails \url{https://guides.rubyonrails.org/v3.2/routing.html}
    \end{itemize}
    
\end{frame}

