%%-------------------------------------------------------------------------------------- Início
\begin{frame}[allowframebreaks, t, fragile]{Ação: Create}
	\begin{itemize}
		\item Cria um novo objeto \alert{post} como os parâmetros que foram passados pelo 
			formulário \alert{new}
		\item Tenta \alert{salvar} o objeto no \alert{banco de dados}
		\item Se sucesso, redireciona para o template \alert{show}
		\item Se insucesso, renderiza o template \alert{new} novamente
		\begin{lstlisting}[style=RubyInputStyle, caption=controllers/posts\_controller.rb]
Class PostsController < ApplicationController
  def create
    @post = Post.new(post_params)

    if @post.save
       redirect_to @post, notice: 'Post foi criado com sucesso!' }
    else
       render :new
    end
  end
	
  private 
  def post_params 
    params.require(:post).permit(:title, :body)
  end 
				
		\end{lstlisting}		
		\begin{itemize}
			\item a linha 20 implementa \alert{strong parameters} para aumentar a segurança da aplicação
		\end{itemize}
	\end{itemize}	
\end{frame}

%%-------------------------------------------------------------------------------------- Início
%\begin{frame}[t, fragile]{Hora de Colocar a Mão na Massa}
%	\begin{itemize}
%		\item Modifique o post\_params para o código abaixo:
%		\begin{lstlisting}[style=RubyInputStyle, caption=controllers/posts\_controller.rb]
%Class PostsController < ApplicationController
%  private 
%    # Never trust parameters from the scary internet, 
%		# only allow the white list through.
%    def post_params 
%      #params.require(:post).permite(:title, :content)
%      params
%    end 
%		\end{lstlisting}		
%		\begin{itemize}
%			\item tente, agora, criar um post (volte agora para o código original).
%		\end{itemize}
%				
%	\end{itemize}	
%\end{frame}

