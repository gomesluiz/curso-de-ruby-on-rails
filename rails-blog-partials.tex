%%-------------------------------------------------------------------------------------- Início
\begin{frame}[allowframebreaks, t, fragile]{Partials}
	\begin{itemize}
		\item Rails encoraja o princípio \alert{DRY}	
		\item O laioute da aplicação é mantida em um único local no arquivo \alert{application.html.erb}
		\item O código comum dos templates ser reutilizado em \alert{múltiplos templates}
		\item Por exemplo, os formulários do \alert{edit} e do \alert{new} - são realmente muito diferentes ?
		\item Partials são similares aos templates regulares, mas ele possuem capacidades mais \alert{refinadas}
		\item Nomes de partials começam com \alert{underscore} (\_) 
		\item Partials são renderizados com \alert{render 'partialname'} (sem underscore)
		%\item \alert{render} também aceita um segundo argumento, um hash com as variáveis locais utilizadas no partial
		%\item Similar a passagem de variáveis locais, o \alert{render} pode receber um objeto
		%\item \alert{$<$\%= render @post \%$>$} renderizara \alert{app/views/posts/\_posts.html.erb} com o conteúdo da variavel @post
\framebreak
		%\item \alert{$<$\%= render @posts \%$>$} renderiza uma coleção e é equivalente a:
		%\begin{lstlisting}[style=RubyInputStyle, caption=controllers/posts\_controller.rb]
		%	<% @posts.each do |posts| %> 
		%		<%= render post %>
		%	<% end %>
		%\end{lstlisting}		
		%\framebreak
		\item \alert{\_form.html.erb}
		\begin{lstlisting}[style=RubyInputStyle, caption=views/posts/\_form.html.erb]
<%= form_with scope: :post, url: posts_path, 
	local: true do |form| %>
	<p>
		<%= form.label :title %><br>
		<%= form.text_field :title %>
	</p>
	
	<p>
		<%= form.label :body %><br>
		<%= form.text_area :body %>
	</p>
	
	<p>
		<%= form.submit %>
	</p>
<% end %>	
<%= link_to 'Back', posts_path %>
		\end{lstlisting}
	\end{itemize}	
\end{frame}


