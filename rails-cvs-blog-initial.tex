%%-------------------------------------------------------------------------------------- Início
\begin{frame}[allowframebreaks, fragile,t]{Hora de Colocar a Mão na Massa}
	\begin{enumerate}
		\item Crie uma conta com usuário e senha no GitHub
		\item Crie o repositório {\bf blog} no GitHub
		
		\item Inicialize os parâmetros de usuário: 
		\begin{lstlisting}[style=BashInputBasicStyle]
			$ git config --global user.name "nome do programador"
			$ git config --global user.email "email do programador"
		\end{lstlisting}
		
		\item Na pasta da aplicação, inicialize o repositório local:
		\begin{lstlisting}[style=BashInputBasicStyle]
			$ git init
		\end{lstlisting}
		
		\item Registre as mudanças realizadas no repositório local:
		\begin{lstlisting}[style=BashInputBasicStyle]
			$ git add .
		\end{lstlisting}

        \item Efetive as mudanças realizadas no repositório local:  		
        \begin{lstlisting}[style=BashInputBasicStyle]
			$ git commit -m "primeiro commit"
		\end{lstlisting}

		\item Associe o repositório local com o repositório remoto utilizando a sua URL:
		\begin{lstlisting}[style=BashInputBasicStyle]
			$ git remote add origin <GITUHB REPOSITORY URL>
		\end{lstlisting}

		\item Registre as alterações no repositório remoto:
		\begin{lstlisting}[style=BashInputBasicStyle]
			$ git push -u origin master
		\end{lstlisting}
	\end{enumerate}
\end{frame}