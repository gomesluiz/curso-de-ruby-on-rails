%%-------------------------------------------------------------------------------------- Início
\section{Aplicação Exemplo}
\begin{frame}[allowframebreaks, fragile,t]{Especificação do Blog App}
	\begin{enumerate}
		\item Blog é uma contração de "weblog", um site de discussão ou troca de informações
		publicado na Web.
		\item Existem dois tipos de participantes: o administrador e o usuário
		\item O administrador do blog deve ser capaz de entrar novas postagens, tipicamente
		em ordem cronológica inversa.
		\item Os usuários devem ser capazes de visitar o blog e escrever comentários sobre
		as postagens.
		\item O administrador do blog deve ser capaz de modificar e ou remover qualquer postagem
		ou comentário.
		\item Os usuários não devem ser capazes de modificar postagens ou comentários de outros usuários.
	\end{enumerate}
\end{frame}

\begin{frame}[allowframebreaks, fragile,t]{Passos Iniciais do Blog App}
	\begin{enumerate}
	    \item Inicie uma janela de terminal e digite no prompt:
	     \begin{lstlisting}[style=BashInputBasicStyle]
		     $ cd 
		     $ rails new blog
	     \end{lstlisting}
	    \item Utilize o gerador \verb|scaffold| para criar os 
	    componentes MVC para as postagens e os comentários
	     \begin{lstlisting}[style=BashInputBasicStyle]
		     $ rails generate scaffold post \ 
			     title:string body:text
	     \end{lstlisting}
	    \item Gere as tabelas \verb|post| e \verb|comment| no banco de dados
	    \begin{lstlisting}[style=BashInputBasicStyle]
		    $ rake db:migrate
	    \end{lstlisting}
	    
	    \item Visualize todas as URLs reconhecidas pela sua aplicação digitando:
	    \begin{lstlisting}[style=BashInputBasicStyle]
	    $ rake routes
	    \end{lstlisting}
	     \item Inicie o servidor web embutido:
	     \begin{lstlisting}[style=BashInputBasicStyle]
	     $ rails s
	     \end{lstlisting}
	     
	     \item Abra uma janela do navegador e digite:
	     \begin{lstlisting}[style=BashInputBasicStyle]
	     $ http://localhost:3000/posts
	     \end{lstlisting}
	\end{enumerate}
  
\end{frame}