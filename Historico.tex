%%\section{Histórico de Evolução}
%%-------------------------------------------------------------------------------------- Início
\begin{frame}[allowframebreaks,fragile,t]{Rails}
  \begin{itemize}
    \item David Heinemeier Hanson \alert{derivou} o Ruby on Rails a partir do BaseCamp --
      uma ferramenta de gestão de projetos da empresa 37Signals.
    \begin{itemize}
	\item a primeira versão de código aberto (em inglês: \textit{open source})foi liberada em 
	  julho de 2004.
	\item mas direitos para que outros desenvolvedores \alert{colaborassem} com o projeto foram liberados
	  em fevereiro de 2005.
    \end{itemize}
    \item Em agosto de 2006, o Ruby on Rails atingiu um \alert{marco importante} quando a Apple dicidiu
      distribuído juntamente com a versão do seu sistema operacional Mac OS X v10.5 "Leopard"
    \begin{itemize}
     \item nesse mesmo no o Rails começou a ganhar muita atenção da comunidade de desenvolvimento web.
    \end{itemize}
    \item Rails é utilizado por diversas companhias, como por exemplo:
    \begin{itemize}
     \item Airbnb, BaseCamp, Disney, GitHub, Hulu, Kickstarter, Shopify e Twitter.
    \end{itemize}

  \end{itemize} 
  \begin{table}\centering\scriptsize
      \begin{tabular}{@{}cl@{}}\toprule
	\textbf{Versão} & \textbf{Data}	\\ \midrule
	1.0 & 13 de dezembro de 2005	\\
	1.2 & 19 de janeiro de 2007	\\
	2.0 & 07 de dezembro de 2007	\\
	2.1 & 01 de junho de 2008	\\
	2.2 & 21 de novembro de 2008	\\ 
	2.3 & 16 de março de 2009	\\
	3.0 & 29 de agosto de 2010	\\
	3.1 & 31 de agosto de 2011	\\
	3.2 & 20 de janeiro de 2012	\\
	4.0 & 25 de junho de 2013	\\
	4.1 & 08 de abril de 2014	\\ \bottomrule
      \end{tabular}
      \caption{Evolução histórica do Ruby on Rails}
    \end{table}  
\end{frame}
