%%-------------------------------------------------------------------------------------- Início
\begin{frame}[allowframebreaks, fragile,t]{Prática 1}
	\begin{enumerate}
		\item Inicie uma janela de terminal e digite no prompt:
			\begin{lstlisting}[style=BashInputBasicStyle]
			$ rails new blog
			\end{lstlisting}

		\item Mude para o diretório da aplicao (RAILS.root)
			\begin{lstlisting}[style=BashInputBasicStyle]
			$ cd blog
			\end{lstlisting}

		\item Execute o servidor web embutido:
		\begin{lstlisting}[style=BashInputBasicStyle]
			$ rails server
		\end{lstlisting}
		
		\item Abra uma janela do navegador e digite:
		\begin{lstlisting}[style=BashInputBasicStyle]
			http://localhost:3000
		\end{lstlisting}

		\item Verifique o conteúdo do arquivo de configuração \verb|database.yml|:
		\begin{lstlisting}[style=BashInputBasicStyle]
			$ cat config/database.yml
		\end{lstlisting}

		\item Crie o banco de dados de desenvolvimento e testes:
		\begin{lstlisting}[style=BashInputBasicStyle]
			$ rake db:create
		\end{lstlisting}

		\item Crie o modelo Post:
		\begin{lstlisting}[style=BashInputBasicStyle]
			$ rails g model Post title:string body:text
		\end{lstlisting}

		\item Implemente modelo Post no banco de dados com {\it migrations}:
		\begin{lstlisting}[style=BashInputBasicStyle]
			$ rake db:migrate
		\end{lstlisting}

		\item Crie o controlador PostsController:
		\begin{lstlisting}[style=BashInputBasicStyle]
			$ rails g controller Posts 		
		\end{lstlisting}

		\item Modifique o arquivo \verb|config/routes.rb| para acrescentar as rotas para
		o recurso posts:
		\begin{lstlisting}[style=RubyInputStyle]
			Rails.application.routes.draw do 
				resources :posts
			end 
		\end{lstlisting}	
		
		\item Execute o comando \verb!rake! para visualizar as rotas para os
		posts:
		\begin{lstlisting}[style=BashInputBasicStyle]
			$ rake routes
		\end{lstlisting}
		
		\item Reinicie o servidor web e acesse a url \url{http:\\localhost:3000/}. Veja o que ocorreu.

	\end{enumerate}
\end{frame}