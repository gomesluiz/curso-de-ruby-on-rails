
\begin{frame}[fragile,t]{Blocos}
  \begin{itemize}
    \item "Pedaço" de código
    \begin{itemize}
      \item escrito entre chaves(${}$) ou entre \alert{do} e \alert{end}
      \item passado para métodos como o \alert{último} parâmetro
    \end{itemize}
    \item \textbf{Convenção} 
    \begin{itemize}
      \item use chaves(${}$) quanto o bloco contém uma linha
      \item 
    \end{itemize}
    \item Usado para tornar o código mais claro
  \end{itemize}
  
  \lstinputlisting[style=RubyInputStyle, caption=parens.rb]{codigos/ruby/03-funcoes-e-metodos/parens.rb}
    
\end{frame}

\begin{frame}[fragile,t]{Parâmetros e Retorno}
  \begin{itemize}
    \item Não é necessário declarar o tipo dos parâmetros
    \item O método pode retornar qualquer valor
    \item O comando \verb!return! é opcional
    \begin{itemize}
      \item o valor da \alert{última linha} executada é retornada 
    \end{itemize}
  \end{itemize}
  
  \lstinputlisting[style=RubyInputStyle, caption=return\_optional.rb]{codigos/ruby/03-funcoes-e-metodos/return_optional.rb}
    
\end{frame}

\begin{frame}[fragile,t]{Nomes de Métodos Expressivos}
  \begin{itemize}
    \item Nomes de métodos podem terminar com:
    \begin{itemize}
    	\item \alert{'?'} - predicate methods
    	\item \alert{'!'} - métodos com efeitos colaterais
    \end{itemize}
    
	\lstinputlisting[style=RubyInputStyle, caption=expressive.rb]{codigos/ruby/03-funcoes-e-metodos/expressive.rb}
  \end{itemize}   
\end{frame}

\begin{frame}[fragile,t]{Argumentos Padrões(Defaults)}
  \begin{itemize}
    \item Métodos podem ter argumentos padrões
    \begin{itemize}
    	\item se o valor é passado, ele é utilizado
    	\item senão, o valor padrão é utilizado
    \end{itemize}
  \end{itemize}  
  \lstinputlisting[style=RubyInputStyle, caption=default\_args.rb]{codigos/ruby/03-funcoes-e-metodos/default_args.rb}
\end{frame}

\begin{frame}[fragile,t]{Quantidade Variável de Argumentos}
  \begin{itemize}
    \item \alert{*} prefixa o parâmetro com quantidade variável de argumentos
  \end{itemize}
  \begin{itemize}
    \item Pode ser utilizado com parâmetros no início, meio e final
  \end{itemize}
  \lstinputlisting[style=RubyInputStyle, caption=splat.rb]{codigos/ruby/03-funcoes-e-metodos/splat.rb}
\end{frame}

\begin{frame}[fragile,t]{Recapitulando}
  \begin{itemize}
    \item \alert{Não há necessidade} de declarar o tipo de parâmetro passado ou retornado (linguagem dinâmica)
    \item \verb!return! é \alert{opcional} - a última linha executável é "retornada"
    \item Permite métodos com \alert{quantidade variável} de argumentos ou argumentos padrão
  \end{itemize}
\end{frame}



