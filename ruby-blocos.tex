
\begin{frame}[allowframebreaks, fragile,t]{Blocos}
  \begin{itemize}
    \item Um "Trecho" de código
    \begin{itemize}
      \item escrito entre chaves(\{\}) ou entre \alert{do} e \alert{end}
      \item passado para métodos como o \alert{último} parâmetro
    \end{itemize}
    \item \textbf{Convenção} 
    \begin{itemize}
      \item use chaves(\{\}) quanto o bloco contém uma linha
      \item use \alert{do} e \alert{end} quando o bloco contém múltiplas linhas
    \end{itemize}
    \item Frequentemente utilizado em \alert{iteração}
  \end{itemize}
  
  \lstinputlisting[style=RubyInputStyle, caption=times.rb]{codigos/ruby/04-blocos/times.rb}
    
\end{frame}

\begin{frame}[fragile,t]{Utilizando Blocos}
  \begin{itemize}
    \item Duas técnicas para utilizar blocos nos métodos
    \item \textbf{Implicitamente:}
    \begin{itemize}
      \item use \verb!block_given?! para checar se o bloco foi passado
      \item use \verb!yield! para \alert{chamar} o bloco
    \end{itemize}
    \item \textbf{Explicitamente:}
    \begin{itemize}
      \item use \verb!&! como prefixo do último parâmetro
      \item use \verb!call! para \alert{chamar} o bloco
    \end{itemize} 
  \end{itemize}   
\end{frame}

\begin{frame}[fragile,t]{Técnica Implícita}
  \begin{itemize}
    \item Necessário checar com \verb!block_given?! 
    \begin{itemize}
    	\item se não uma excessão será lançada
    \end{itemize}
    
	\lstinputlisting[style=RubyInputStyle, caption=implicit\_blocks.rb]{codigos/ruby/04-blocos/implicit_blocks.rb}
  \end{itemize}   
\end{frame}

\begin{frame}[fragile,t]{Técnica Explícita}
  \begin{itemize}
    \item Necessário checar com \verb!nil?! 
    
	\lstinputlisting[style=RubyInputStyle, caption=implicit\_blocks.rb]{codigos/ruby/04-blocos/explicit_blocks.rb}
  \end{itemize}   
\end{frame}

\begin{frame}[fragile,t]{Recapitulando}
  \begin{itemize}
    \item Blocos são apenas \alert{trechos} de códigos que podem ser passados para métodos
    \item Tanto explicitamente quanto implicitamente
  \end{itemize}
\end{frame}



