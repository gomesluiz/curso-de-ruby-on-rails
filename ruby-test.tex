
\begin{frame}[fragile,t]{Minitest}
  \begin{itemize}
    \item Frameworks para automatização de testes Ruby na linguagem Ruby.
  \end{itemize}
\end{frame}


\begin{frame}[fragile,t]{Instalação}    
  \begin{lstlisting}[style=RubyInputStyle]
      $ gem install minitest
  \end{lstlisting}
\end{frame}

\begin{frame}[fragile,t]{Acesso a Variáveis de Instância}    
  \begin{lstlisting}[style=RubyInputStyle]
    require 'minitest/autorun'
    require_relative './temperature'
    
  \end{lstlisting}
\end{frame}

\begin{frame}[allowframebreaks, fragile,t]{Métodos e Variáveis de Classe}    
  \begin{itemize}
    \item Use \alert{self} para definir métodos de classe
    \item Variáveis de classe começam com \alert{@@}
  \end{itemize}  
  \lstinputlisting[style=RubyInputStyle, caption=class\_methods\_and\_variables.rb]{codigos/ruby/11-more-classes/class_methods_and_variables.rb}
 
\end{frame}

\begin{frame}[allowframebreaks, fragile,t]{Herança de Classes}    
  \lstinputlisting[style=RubyInputStyle, caption=inheritance.rb]{codigos/ruby/11-more-classes/inheritance.rb}
 \end{frame}
%-------------------------------------------------------------------------------------- Início
\begin{frame}[fragile, t]{Visibilidade de Métodos em Ruby}
  \begin{enumerate}
    \item Todos os atributos são \alert{privado} por padrão e Todos os métodos
    são públicos por padrão.
    \item \alert{private} e \alert{protected} podem ser utilizados
    para mudar a visibilidade padrão de \alert{métodos}
  \end{enumerate}
  %TODO: módulos e exemplos
\end{frame}
%-------------------------------------------------------------------------------------- Início
\begin{frame}[allowframebreaks, fragile, t]{Hora de Colocar as Mãos na Massa}
  \begin{enumerate}
    \item Elabore na linguagem Ruby os códigos para os seguintes requisitos:
    \begin{enumerate}
          \item Escreva a classe \textbf{Sobremesa} com \textit{getters} e \textit{setters} para os atributos \texttt{nome} e \texttt{calorias}. O construtor dessa classe deverá receber como parâmetros \texttt{nome} e \texttt{calorias}.      
          \item Defina as operações de instância \texttt{ehSaudavel}, que retorna \texttt{true} se e somente se a sobremesa tem menos de 200 calorias, e \texttt{ehDeliciosa}, que retorna \texttt{true} para todas as sobremesas.
          \item Crie a classe \textbf{GeleiaEmCompota} que herdará da classe \textbf{Sobremesa}. O seu construtor deverá aceitar um único argumento denominado \texttt{sabor};  a sua quantidade padrão de \texttt{calorias} é 5 e seu \texttt{nome} deverá ser precedido de "Geléia em Compota de ", por exemplo, "Geléia em Compota de Morango".
          \item Inclua um \textit{getter} and \textit{setter} para o atributo \texttt{sabor}.   
          \item Modifique a operação \texttt{ehDeliciosa} para retornar \textbf{false} se o sabor é \texttt{alcaçuz} e true para todos os outros sabores. O comportamento dessa operação para sobremesas que não são geléias em compotas não devem ser alterados.
    \end{enumerate}  
    \item Refatore o jogo de adivinhação para utilizar os mecanismos da orientação a 
    objetos.  
  \end{enumerate}
\end{frame}

\begin{frame}[fragile,t]{Recapitulando}
  \begin{itemize}
    \item Objetos são criados com \alert{new}
    \item Utilize o \alert{attr\_} para criar getters/setters
    \item Não se esqueça do \alert{self} quando necessário
    \item Variáveis de classe são definidas com \alert{@@}
  \end{itemize}
\end{frame}



