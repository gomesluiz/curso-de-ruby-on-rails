\begin{frame}[allowframebreaks, t, fragile]{I2 - Hora de Colocar a Mão na Massa}
	\begin{itemize}
			
		\item Modifique o arquivo \verb|app/models/post.rb| para exigir 
			que o usuário digite o título e o texto do blog:
		\begin{lstlisting}[style=RubyInputStyle]
			class Post < ApplicationRecord
			  validates_presence_of :title, :body
			end
		\end{lstlisting}
		
		\item Modifique o arquivo \verb|app/models/comment.rb| para exigir 
		que o usuário digite texto do comentário blog:
		\begin{lstlisting}[style=RubyInputStyle]
			class Comment < ApplicationRecord
		    validates_presence_of :body
			end
		\end{lstlisting}
		
		\item Inclua as seguintes instruções no início do arquivo \verb|_form.html.rb| para
		 mostrar as mensagens de validação:
		\begin{lstlisting}[style=RubyInputStyle, caption=app/views/posts/\_form.html.erb]
<% if @post.errors.any? %>
  <div class="alert alert-danger" role="alert">
    <h4 class="alert-heading">Request failed!</h4>
    <ul>
      <% @post.errors.full_messages.each do |message| %>
        <li><%= message %></li>
      <% end %>
    </ul>
    <hr>
    <p class="mb-0">Please, correct issues and try again.</p>
  </div>
<% end %>

		\end{lstlisting}
		
%		\item \alert{Reinicie} a console do Rails tente criar um Post e um Comment
%\begin{lstlisting}[style=BashInputBasicStyle]
%		irb(main):005:0> p1 = Post.new
%		irb(main):006:0> p1.body="Tem algo errado..."
%		irb(main):007:0> p1.save
%		irb(main):008:0> Post.all
%		irb(main):009:0> c1 = Comment.new
%		irb(main):010:0> c1.save
%		irb(main):011:0> Comment.all
%end{lstlisting}	
	\end{itemize}
\end{frame}