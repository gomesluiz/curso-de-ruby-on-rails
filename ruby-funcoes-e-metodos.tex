
\begin{frame}[fragile,t]{Funções e Métodos}
  \begin{itemize}
    \item Tecnicamente, uma \alert{função} é definida \alert{fora} de uma classe
    \item Um \alert{método} é definido dentro de uma classe
    \item Em Ruby, \alert{toda} função/método é pertence a pelo menos uma classe
    \begin{itemize}
      \item nem sempre explicitamente escrito em uma classe
    \end{itemize}
  \end{itemize}
  \framebreak   
  \begin{center}
    Conclusão: Toda \alert{função} é na verdade um \alert{método} em Ruby  
  \end{center}
\end{frame}

\begin{frame}[fragile,t]{Métodos}
  \begin{itemize}
    \item Parênteses são \alert{opcionais}
    \begin{itemize}
      \item tanto para definição quanto para a chamada do método
    \end{itemize}
    \item Usado para tornar o código mais claro
  \end{itemize}
  \begin{block}{parens.rb}
  	\lstinputlisting[style=RubyInputStyle]{codigos/ruby/04-funcoes-e-metodos/parens.rb}
  \end{block}
  
    
\end{frame}

\begin{frame}[fragile,t]{Parâmetros e Retorno}
  \begin{itemize}
    \item Não é necessário declarar o tipo dos parâmetros
    \item O método pode retornar qualquer valor
    \item O comando \verb!return! é opcional
    \begin{itemize}
      \item o valor da \alert{última linha} executada é retornada 
    \end{itemize}
  \end{itemize}
  
  \lstinputlisting[style=RubyInputStyle, caption=return\_optional.rb]{codigos/ruby/04-funcoes-e-metodos/return_optional.rb}
    
\end{frame}

\begin{frame}[fragile,t]{Nomes de Métodos Expressivos}
  \begin{itemize}
    \item Nomes de métodos podem terminar com:
    \begin{itemize}
    	\item \alert{'?'} - métodos com retorno booleano
    	\item \alert{'!'} - métodos com efeitos colaterais
    \end{itemize}
    
	\lstinputlisting[style=RubyInputStyle, caption=expressive.rb]{codigos/ruby/04-funcoes-e-metodos/expressive.rb}
  \end{itemize}   
\end{frame}

\begin{frame}[fragile,t]{Argumentos Padrões(Defaults)}
  \begin{itemize}
    \item Métodos podem ter argumentos padrões
    \begin{itemize}
    	\item se o valor é passado, ele é utilizado
    	\item senão, o valor padrão é utilizado
    \end{itemize}
  \end{itemize}  
  \lstinputlisting[style=RubyInputStyle, caption=default\_args.rb]{codigos/ruby/04-funcoes-e-metodos/default_args.rb}
\end{frame}

\begin{frame}[fragile,t]{Quantidade Variável de Argumentos}
  \begin{itemize}
    \item \alert{*} prefixa o parâmetro com quantidade variável de argumentos
  \end{itemize}
  \begin{itemize}
    \item Pode ser utilizado com parâmetros no início, meio e final
  \end{itemize}
  \lstinputlisting[style=RubyInputStyle, caption=splat.rb]{codigos/ruby/04-funcoes-e-metodos/splat.rb}
\end{frame}
%-------------------------------------------------------------------------------------- Início
\begin{frame}[fragile,t]{Exercícios}
  \begin{enumerate}
    \item Escreva uma função que receba como parâmetro um custo retorna a taxa de entrega
		de acordo com a seguinte regra:
		\begin{itemize}
			\item a taxa de entrega será igual a 10.00 se o custo for menor do que 25.00;
			\item a taxa será igual a 20.00 se o custo for menor do que 100.00;
			\item a taxa a taxa será igual a 30.00 se o custo for menor do que 200.00;
			\item a taxa será igual a 35.00 se o custo for maior ou igual do que 200.00;
		\end{itemize} 
  \end{enumerate}
\end{frame}

\begin{frame}[fragile,t]{Recapitulando}
  \begin{itemize}
    \item \alert{Não há necessidade} de declarar o tipo de parâmetro passado ou retornado (linguagem dinâmica)
    \item \verb!return! é \alert{opcional} - a última linha executável é "retornada"
    \item Permite métodos com \alert{quantidade variável} de argumentos ou argumentos padrão
  \end{itemize}
\end{frame}



