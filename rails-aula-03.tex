\documentclass[t, 				             
			   final,
			   12pt, 				         
			   xcolor={usenames,dvipsnames}, 
			   table]{beamer}

% pacotes utilizados.
\usepackage[alf]{abntex2cite}
\usepackage{amsmath}
\usepackage[brazil]{babel}
\usepackage{booktabs}
\usepackage{caption}
\usepackage[utf8]{inputenc}
\usepackage{listings}
\usepackage{multicol}
\usepackage{multirow}
\usepackage{todo}


% configuração do tema
\usetheme[pageofpages=de,
          bullet=square,			
          titleline=false,				
          alternativetitlepage=true,			
          titlepagelogo=imagens/logo-puc,	
          watermarkheight=70px,		
          watermarkheightmult=4	
          ]{Torino}

\setbeamertemplate{sections/subsections in toc}[square]
\setbeamertemplate{bibliography item}[default]

\usecolortheme{freewilly}


\definecolor{javared}{rgb}{0.6,0,0} % for strings
\definecolor{javagreen}{rgb}{0.25,0.5,0.35} % comments
\definecolor{javapurple}{rgb}{0.5,0,0.35} % keywords
\definecolor{javadocblue}{rgb}{0.25,0.35,0.75} % javadoc
 
\lstset{}

\lstdefinestyle{BashInputBasicStyle}{
	language=bash,
	basicstyle=\normalsize\ttfamily,
	columns=fullflexible,
	tabsize=2,
	showstringspaces=false,
	frame=single,
	inputencoding=utf8,
	rulecolor=\color{gray}
}

\lstdefinestyle{BashInputStyle}{
  language=bash,
  basicstyle=\normalsize\ttfamily,
  numbers=left,
  numberstyle=\tiny,
  numbersep=2pt,
  frame=tb,
  columns=fullflexible,
  tabsize=2,
  showstringspaces=false,
  commentstyle=\color{gray},
  inputencoding=utf8,
  rulecolor=\color{gray}
}

\lstdefinestyle{RubyInputStyle}{
    language=ruby,
    basicstyle=\scriptsize\ttfamily,
    keywordstyle=\color{javapurple},
    identifierstyle=\color{black},
    commentstyle=\color{javagreen},
	stringstyle=\color{blue},
    showstringspaces=false,
    numbers=left,
    numberstyle=\color{gray}\tiny,
    tabsize=3,
    extendedchars=\true,
    inputencoding=utf8,
%   frame=single, 
    columns=fixed,
    backgroundcolor=\color{red!32!green!33!blue!5}
}    
%  language=ruby,
%  basicstyle=\normalsize\ttfamily,
%  keywordstyle=\color{OrangeRed},
%  identifierstyle=\color{Turquoise},
%  commentstyle=\color{gray},
%  stringstyle=\color{YellowOrange},
%  numbers=left,
%  numberstyle=\tiny,
%  numbersep=2pt,
%  frame=tb,
%  columns=fullflexible,
%  backgroundcolor=\color{white!80},
%  linewidth=0.9\linewidth,
%  tabsize=2,
%  showstringspaces=false
%  inputencoding=utf8


\lstdefinestyle{JavaInputStyle}{
	language=Java,
	basicstyle=\ttfamily,
	keywordstyle=\color{javapurple}\bfseries,
	stringstyle=\color{javared},
	commentstyle=\color{javagreen},
	morecomment=[s][\color{javadocblue}]{/**}{*/},
	numbers=left,
	numberstyle=\tiny\color{black},
	numbersep=10pt,
	tabsize=2,
	showspaces=false,
	showstringspaces=false,
	frame=tb,
	columns=fullflexible,
	backgroundcolor=\color{white!80},
	linewidth=0.9\linewidth,
	inputencoding=utf8
}

\begin{document}
	\author{Luiz Alberto Ferreira Gomes}
\title{Aula 03: Ruby On Rails}
\subtitle{Laboratório de Projeto de Sistemas}
\institute{Curso de Ciência da Computação}
\date{\today}

	\include{Capa}
	\include{Agenda}  	

    \section{Nova Postagem}
        %%-------------------------------------------------------------------------------------- Início
\begin{frame}[allowframebreaks, t, fragile]{Ação: New}
	\begin{figure}[h!]
		\centering
		\includegraphics[width=0.75\textwidth]{imagens/mvc-action-new.jpg}
	\end{figure}
	\framebreak
	\begin{itemize}
		\item Um novo objeto \alert{@post} da classe \alert{Post} é instanciado 
		\item Procura pela visão \alert{new.html.erb} para renderizar a resposta
		\begin{lstlisting}[style=RubyInputStyle, caption=app/controllers/posts\_controller.rb]
Class PostsController < ApplicationController
  def new
    @post = Post.new 
  end 
end
		\end{lstlisting}
	\end{itemize}
\end{frame}

\begin{frame}[allowframebreaks, t, fragile]{Visão: New}
	\begin{itemize}
		\item Reinicie o servidor web e acesse a url \url{http:\\localhost:3000/posts/new}. Veja o erro que ocorreu.
		\item Implemente a visão \alert{new.html.erb}:
		\begin{lstlisting}[style=RubyInputStyle, caption=views/posts/new.html.erb]
<h1>Novo Post</h1>
<%= form_with scope: :post, url: posts_path, 
	local: true do |form| %>
	<p>
		<%= form.label :title %><br>
		<%= form.text_field :title %>
	</p>
	
	<p>
		<%= form.label :body %><br>
		<%= form.text_area :body %>
	</p>
	
	<p>
		<%= form.submit %>
	</p>
<% end %>	
		\end{lstlisting}
	\end{itemize}	
\end{frame}
        %%-------------------------------------------------------------------------------------- Início
\begin{frame}[allowframebreaks, t, fragile]{Ação: Create}
  \begin{figure}[h!]
		\centering
		\includegraphics[width=0.75\textwidth]{imagens/mvc-action-create.jpg}
	\end{figure}
	\framebreak
  \begin{itemize}
		\item Um novo objeto \alert{@post} da classe \alert{Post} é criado com os parâmetros que foram passados pelo 
			formulário \alert{new}
		\item Tenta \alert{salvar} o objeto \alert{@post} no \alert{banco de dados}
%		\item Se sucesso, redireciona para o template \alert{show}
%		\item Se insucesso, renderiza o template \alert{new} novamente
		\begin{lstlisting}[style=RubyInputStyle, caption=controllers/posts\_controller.rb]
class PostsController < ApplicationController
  def new
      @post = Post.new
  end

  def create 
      @post = Post.new(post_params)
      
      @post.save
      redirect_to @post
  end

private 
  def post_params 
    params.require(:post).permit(:title, :body)
  end
end          
		\end{lstlisting}		
		\begin{itemize}
			\item a linha 15 implementa \alert{strong parameters} para aumentar a segurança da aplicação
    \end{itemize}
    \item Como a ação {\bf Show} ainda não foi implementada, ocorrerá uma erro quando
    o botão {\bf Submit} for pressionado.
	\end{itemize}	
\end{frame}
    \section{Exibir Postagem}
        %%-------------------------------------------------------------------------------------- Início
\begin{frame}[allowframebreaks, t, fragile]{Ação: Show}
	\begin{figure}[h!]
		\centering
		\includegraphics[width=0.75\textwidth]{imagens/mvc-action-show.jpg}
	\end{figure}
	\framebreak
	\begin{itemize}
		\item Recupera \alert{uma} postagem específica no parâmetro \alert{id} passado como parte da URL
		\item (Implicitamente) procura pelo \alert{show.html.erb} para renderizar a resposta
		\begin{lstlisting}[style=RubyInputStyle, caption=controllers/posts\_controller.rb]
class PostsController < ApplicationController
	def show 
		@post = Post.find(params[:id])
	end

	def new
		@post = Post.new
	end

	def create 
		@post = Post.new(post_params)
		
		@post.save
		redirect_to @post
	end
private 
	def post_params 
		params.require(:post).permit(:title, :body)
	end
end
		\end{lstlisting}
	\end{itemize}	
\end{frame}

%%-------------------------------------------------------------------------------------- Início
\begin{frame}[allowframebreaks, t, fragile]{Visão: Show}
	\begin{itemize}
		\item Implemente a visão \alert{show.html.erb}:
		\begin{lstlisting}[style=RubyInputStyle, caption=views/posts/show.html.erb]
<p>
	<strong>Title:</strong>
	<%= @post.title %>
</p>
<p>
<strong>Body:</strong>
	<%= @post.body %>
</p>
		\end{lstlisting}		
	\end{itemize}	
\end{frame}
    \section{Atualizar Postagem}
        %%-------------------------------------------------------------------------------------- Início
\begin{frame}[t, fragile]{Ação: Edit}
	\begin{itemize}
		\item Recupera uma postagem específica no parâmetro \alert{id} passado como parte da URL
		\item (Implicitamente) procura pelo \alert{edit.html.erb} para renderizar a resposta
		\begin{lstlisting}[style=RubyInputStyle, caption=controllers/posts\_controller.rb]
Class PostsController < ApplicationController
  def edit 
    @post = Post.find(params[:id])
  end 
		\end{lstlisting}		
	\end{itemize}	
\end{frame}
    \section{Reuso com Partials}
        %%-------------------------------------------------------------------------------------- Início
\begin{frame}[allowframebreaks, t, fragile]{Partials}
	\begin{itemize}
		\item Rails encoraja o princípio \alert{DRY}	
		\item O laioute da aplicação é mantida em um único local no arquivo \alert{application.html.erb}
		\item O código comum dos templates ser reutilizado em \alert{múltiplos templates}
		\item Por exemplo, os formulários do \alert{edit} e do \alert{new} - são realmente muito diferentes ?
		\item Partials são similares aos templates regulares, mas ele possuem capacidades mais \alert{refinadas}
		\item Nomes de partials começam com \alert{underscore} (\_) 
		\item Partials são renderizados com \alert{render 'partialname'} (sem underscore)
		%\item \alert{render} também aceita um segundo argumento, um hash com as variáveis locais utilizadas no partial
		%\item Similar a passagem de variáveis locais, o \alert{render} pode receber um objeto
		%\item \alert{$<$\%= render @post \%$>$} renderizara \alert{app/views/posts/\_posts.html.erb} com o conteúdo da variavel @post
\framebreak
		%\item \alert{$<$\%= render @posts \%$>$} renderiza uma coleção e é equivalente a:
		%\begin{lstlisting}[style=RubyInputStyle, caption=controllers/posts\_controller.rb]
		%	<% @posts.each do |posts| %> 
		%		<%= render post %>
		%	<% end %>
		%\end{lstlisting}		
		%\framebreak
		\item \alert{\_form.html.erb}
		\begin{lstlisting}[style=RubyInputStyle, caption=views/posts/\_form.html.erb]
<%= form_with scope: :post, url: posts_path, 
	local: true do |form| %>
	<p>
		<%= form.label :title %><br>
		<%= form.text_field :title %>
	</p>
	
	<p>
		<%= form.label :body %><br>
		<%= form.text_area :body %>
	</p>
	
	<p>
		<%= form.submit %>
	</p>
<% end %>	
<%= link_to 'Back', posts_path %>
		\end{lstlisting}
	\end{itemize}	
\end{frame}



        %%-------------------------------------------------------------------------------------- Início
\begin{frame}[allowframebreaks, t, fragile]{View: Edit}
	\begin{itemize}
		\item \alert{edit.html.erb}:
		\begin{lstlisting}[style=RubyInputStyle, caption=view/posts/edit.html.erb]
			<h1>Edit Post</h1>
			<%= render 'form' %>
		\end{lstlisting}		
	\end{itemize}	
\end{frame}
        %%-------------------------------------------------------------------------------------- Início
\begin{frame}[allowframebreaks, t, fragile]{Ação: Update}
	\begin{itemize}
		\item Recupera um objeto \alert{post} utilizando o parâmetro \alert{id}	
		\item Atualiza o objeto \alert{post} com os parâmetros que foram passados pelo
			formulário \alert{edit}
		\item Tenta \alert{atualizar} o objeto no \alert{banco de dados}
		%\item Se sucesso, redireciona para o template \alert{show}
		%\item Se insucesso, renderiza o template \alert{edit} novamente
		\begin{lstlisting}[style=RubyInputStyle, caption=posts\_controller.rb]
Class PostsController < ApplicationController
  def update
    @post = Post.find(params[:id])
	@post.update(post_params)
	redirect_to @post
  end
  ...
  	\end{lstlisting}				
	\end{itemize}	
\end{frame}



    \section{Visão Destroy}
        %%-------------------------------------------------------------------------------------- Início
\begin{frame}[allowframebreaks, t, fragile]{Ação: Destroy}
	\begin{itemize}
		\item Remove uma postagem específica pelo parâmetro \alert{id} passado como parte da URL
		\begin{lstlisting}[style=RubyInputStyle, caption=posts\_controller.rb]
Class PostsController < ApplicationController
  def destroy
    @post = Post.find(params[:id])
	@post.destroy
	redirect_to posts_path
  end 
		\end{lstlisting}		
	\end{itemize}	
\end{frame}
    \section{Controle de Versão}
        %%-------------------------------------------------------------------------------------- Início
\begin{frame}[allowframebreaks, fragile,t]{Hora de Colocar a Mão na Massa}
	\begin{enumerate}
		\item Registre as mudanças realizadas no repositório local:
		\begin{lstlisting}[style=BashInputBasicStyle]
			$ git add .
		\end{lstlisting}

        \item Efetive as mudanças realizadas no repositório local:  		
        \begin{lstlisting}[style=BashInputBasicStyle]
			$ git commit -m "#1: manutencao de posts"
		\end{lstlisting}

		\item Registre as alterações no repositório remoto:
		\begin{lstlisting}[style=BashInputBasicStyle]
			$ git push -u origin master
		\end{lstlisting}
	\end{enumerate}
\end{frame}
    
    \section{Para Saber Mais}
        %%-------------------------------------------------------------------------------------- Início
\begin{frame}[fragile,t]{Para Saber Mais}
  \begin{itemize}
    \item \url{https://www.ruby-lang.org/en/}
    \begin{itemize}
     \item referência oficial da linguagem Ruby onde a toda a sua documentação está disponível
	para ser consultada.
    \end{itemize}

    \item \url{http://rubyonrails.org/}
    \begin{itemize}
     \item referência oficial do framework Rails onde a toda a sua documentação está disponível
	para ser consultada.
    \end{itemize}
    
    \item \url{http://www.codecademy.com/pt/tracks/ruby}
    \begin{itemize}
     \item curso iterativo em portugês sobre a linguagem Ruby.
    \end{itemize}

	\item \url{https://gorails.com/setup/ubuntu/16.04}
	\begin{itemize}
		\item guia para instalação do Ruby on Rails no Ubuntu e no Mac OSX.
	\end{itemize}
  \end{itemize}
  
  
\end{frame}
\end{document}
