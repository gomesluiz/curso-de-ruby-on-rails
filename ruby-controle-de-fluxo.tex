
\begin{frame}[allowframebreaks, fragile,t]{Controle de Fluxo}
  \begin{itemize}
    \item \verb!if ... elsif ... else!
    \item \verb!case!
    \item \verb!unless!
  \end{itemize}   
\pagebreak
  \begin{itemize}
    \item \verb!if, unless, elsif, else!
    \item Não existe a necessidade de uso de parênteses ou chaves
    \item Utilize a instrução \verb!end! no final do bloco
  \end{itemize}   
  \begin{columns}
    \begin{column}{0.5\textwidth}
    		\lstinputlisting[style=RubyInputStyle, caption=if.rb]{codigos/ruby/02-fluxo-de-controle/if.rb}  
    \end{column}
    \begin{column}{0.5\textwidth}
    		\lstinputlisting[style=RubyInputStyle, caption=unless.rb]{codigos/ruby/02-fluxo-de-controle/unless.rb}  
    \end{column}
  \end{columns}
\end{frame}

\begin{frame}[fragile,t]{True / False}
  \begin{itemize}
    \item false e nil são booleanos FALSOS
    \item Todo o restante é VERDADEIRO
	\lstinputlisting[style=RubyInputStyle, caption=true\_false.rb]{codigos/ruby/02-fluxo-de-controle/true_false.rb}
  \end{itemize}   
\end{frame}


\begin{frame}[fragile,t]{Recapitulando}
  \begin{itemize}
    \item Existe muitas opções de fluxo de controle
    \item A formato em um linha é muito expressiva
    \item Exceto \verb!nil! e \verb!false!, os demais valores são verdadeiros.
  \end{itemize}
\end{frame}



