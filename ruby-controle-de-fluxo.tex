%-------------------------------------------------------------------------------------- Início
\begin{frame}[allowframebreaks, fragile,c]{Controle de Fluxo}
  \begin{center}
    \Large \verb!if ... elsif ... else! \\ \verb!unless! \\ \verb!case! 
  \end{center}   
\framebreak
  \begin{itemize}
    \item \alert{Não} existe a necessidade de uso de \alert{parênteses} ou \alert{chaves}
    \item \alert{Utilize} a instrução \verb!end! no final do bloco
  \end{itemize}   
  \begin{columns}
    \begin{column}{0.5\textwidth}
    		\lstinputlisting[style=RubyInputStyle, caption=if.rb]{codigos/ruby/02-fluxo-de-controle/if.rb}  
    \end{column}
    \begin{column}{0.5\textwidth}
    		\lstinputlisting[style=RubyInputStyle, caption=unless.rb]{codigos/ruby/02-fluxo-de-controle/unless.rb}  
    \end{column}
  \end{columns}
\framebreak
    		\lstinputlisting[style=RubyInputStyle, caption=case\_1.rb]{codigos/ruby/02-fluxo-de-controle/case_1.rb}  
\framebreak
    		\lstinputlisting[style=RubyInputStyle, caption=case\_2.rb]{codigos/ruby/02-fluxo-de-controle/case_2.rb}  
\end{frame}
%-------------------------------------------------------------------------------------- Início
\begin{frame}[fragile,t]{Operadores Lógicos (em ordem de precedência)}
	\begin{table}[tp] 	
		\setlength{\tabcolsep}{8pt}
    \setlength{\extrarowheight}{2pt}    
		\begin{tabular}{p{2.5cm}l}
    	\toprule
      $<=, <, >, >=$ &  Comparação\\
      $==, !=$ & Igual ou diferente  \\
      $\&\&$ &  Conectivo \textbf{e} \\
      $||$ & Conectivo \textbf{ou} \\
	    \bottomrule
		\end{tabular}
	\end{table}   
\end{frame}
%-------------------------------------------------------------------------------------- Início
\begin{frame}[fragile,t]{True e False}
  \begin{itemize}
    \item \alert{false} e \alert{nil} são booleanos \alert{FALSOS}
    \item Todo o restante é \alert{VERDADEIRO}
	\lstinputlisting[style=RubyInputStyle, caption=true\_false.rb]{codigos/ruby/02-fluxo-de-controle/true_false.rb}
  \end{itemize}   
\end{frame}

\begin{frame}[allowframebreaks,fragile,t]{Hora de Colocar as Mão na Massa}
  \begin{enumerate}
    \item Escreva um \textit{script} Ruby leia o nome de um jogador do jogo de adivinha é um 
    palpite, ambos digitados pelo teclado. Após isto, o script deverá apresentar os valores lidos e se 
    o usuário acertou o número secreto ou não.
  \end{enumerate}
  \framebreak
\end{frame}
%-------------------------------------------------------------------------------------- Início
% \begin{frame}[fragile,t]{Exercícios}
%  \begin{enumerate}
%    \item Digite as seguintes expressões no IRB e verifique os resultados.
%  \end{enumerate}
%	\begin{lstlisting}[style=RubyInputStyle]
%		(32 * 4) >= 129
%		false != !true
%		true == 4
%		false == (847 == '847')
%		(!true || (!(100 / 5) == 20) || ((328 / 4) == 82) || false 
%	\end{lstlisting}
%\end{frame}
%-------------------------------------------------------------------------------------- Início
\begin{frame}[fragile,t]{Recapitulando}
  \begin{itemize}
    \item Existe muitas opções de fluxo de controle
    \item A formato em um linha é muito expressiva
    \item Exceto \verb!nil! e \verb!false!, os demais valores são verdadeiros.
  \end{itemize}
\end{frame}



