
\begin{frame}[allowframebreaks, fragile,t]{Controle de Fluxo}
  \begin{itemize}
    \item \verb!if / elsif / else!
    \item \verb!case!
    \item \verb!until / unless!
    \item \verb!while / for!
  \end{itemize}   
\pagebreak
  \begin{itemize}
    \item \verb!if, unless, elsif, else!
    \item Não existe a necessidade de uso de parênteses ou chaves
    \item Utilize a instrução \verb!end! no final do bloco
  \end{itemize}   
  \begin{columns}
    \begin{column}{0.5\textwidth}
      \begin{lstlisting}[style=RubyInputStyle, caption=if.rb]
    a = 5 
    if  a == 3                    
      puts "a igual a 3"                  
    elsif a == 5 
      puts "a igual a 5"                   
    else                              
      puts "a nao e igual 3 or 5"
    end 
      \end{lstlisting}  
    \end{column}
    \begin{column}{0.5\textwidth}  %%<--- here
      \begin{lstlisting}[style=RubyInputStyle, caption=unless.rb]
    a = 5 
    unless a == 6
      puts "a nao e 6"
    end 
      \end{lstlisting}
    \end{column}
  \end{columns}
\pagebreak
  \begin{itemize}
    \item \verb!while, until!
  \end{itemize}   
  \begin{columns}
    \begin{column}{0.5\textwidth}
      \begin{lstlisting}[style=RubyInputStyle, caption=while.rb]
    a = 10
    while a > 9
      puts a 
      a -= 1 
    end 
      \end{lstlisting}  
    \end{column}
    \begin{column}{0.5\textwidth}  %%<--- here
      \begin{lstlisting}[style=RubyInputStyle, caption=until.rb]
    a = 9
    until
      puts "a nao e 6"
    end 
      \end{lstlisting}
    \end{column}
  \end{columns}
\pagebreak
  \begin{itemize}
    \item \verb!if, unless, while, until! - na mesma linha da instrução
  \end{itemize}   
      \begin{lstlisting}[style=RubyInputStyle, caption=if\_uma\_linha.rb]
    a = 5
    b = 0
    puts "Em uma linha" if a == 5 and b == 0
      \end{lstlisting}  
      \begin{lstlisting}[style=RubyInputStyle, caption=while\_uma\_linha.rb]
    conta = 2 
    conta *= 2 while conta < 100
    puts conta
      \end{lstlisting}
\end{frame}

\begin{frame}[fragile,t]{True / False}
  \begin{itemize}
    \item false e nil são booleanos FALSOS
    \item Todo o restante é VERDADEIRO
	\lstinputlisting[style=RubyInputStyle, caption=true\_false.rb]{codigos/ruby/02-fluxo-de-controle/true_false.rb}
  \end{itemize}   
\end{frame}

\begin{frame}[allowframebreaks, fragile,t]{For Loop}
  \begin{itemize}
    \item Dificilmente empregado
    \item \verb!each/times! é preferível
  \end{itemize}
  
  \lstinputlisting[style=RubyInputStyle, caption=for\_loop.rb]{codigos/ruby/02-fluxo-de-controle/for_loop.rb}

\end{frame}


\begin{frame}[fragile,t]{Recapitulando}
  \begin{itemize}
    \item Existe muitas opções de fluxo de controle
    \item A formato em um linha é muito expressiva
    \item Exceto \verb!nil! e \verb!false!, os demais valores são verdadeiros.
  \end{itemize}
\end{frame}



