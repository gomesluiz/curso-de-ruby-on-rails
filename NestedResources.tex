\begin{frame}[allowframebreaks, t, fragile]{I13 - Hora de Colocar a Mão na Massa}
	\begin{itemize}
		\item Modifique o arquivo de rotas para aninhar os comentários às postagens e reinicie o servidor:
		\begin{lstlisting}[style=RubyInputStyle, caption=config/routes.rb]
Rails.application.routes.draw do
  resources :posts do 
    resources :comments
  end 
end 
		\end{lstlisting}

%		\item Crie um novo \verb|post| e salve no banco de dados:
%		\begin{lstlisting}[style=BashInputBasicStyle]
%			irb(main):005:0> p1 = Post.new
%			irb(main):006:0> p1.title="Whatsapp bloqueado"
%			irb(main):007:0> p1.body="A justica bloqueou o Whatsapp..."
%			irb(main):008:0> p1.save
%		\end{lstlisting}
%		
%		\item Crie um novo \verb|comment| e o vincule a um \verb|post|:
%		\begin{lstlisting}[style=BashInputBasicStyle]
%			irb(main):005:0> c1 = Comment.new
%			irb(main):006:0> c1.body="O que fazer agora ???"
%			irb(main):007:0> c1.save
%			irb(main):008:0> p1.comments << c1
%		\end{lstlisting}
		
%		\item Digite no navegador no endereço \url{http://localhost:3000/posts/id/comments}. Onde 
%		o \alert{id} é o id de um post qualquer.
%				
%		\item Agora crie um novo blog e digite novamente \url{http://localhost:3000/posts/id/comments}. Onde 
%		o \alert{id} do blog que acabou de ser criado.(\alert{Temos um problema})
				
		\item Modifique o código do template \alert{views/posts/show.html.erb}. Insira o código abaixo do parágrafo do body.
		\begin{lstlisting}[style=RubyInputStyle]
<h2>Comments</h2>
<div id="comments">
  <% @post.comments.each do |comment| %>
	<p>
  <strong>Posted <%= time_ago_in_words(comment.created_at) %></strong>
  <%= h(comment.body) %>
  </p>
  <% end %>
</div>
		\end{lstlisting}

	\item Agora no navegador visualize uma postagem que tenha comentários.
	\item Acrescente o código a seguir logo abaixo do código anterior no arquivo \alert{views/posts/show.html.erb}:
	\begin{lstlisting}[style=RubyInputStyle]
<%= form_for([@post, Comment.new]) do |f| %>
<p>
<%= f.label :body, "New Comment" %><br>
<%= f.text_area :body %>
</p>
<p>
<%= f.submit "Add Comments" %>
</p>
<% end %>
\end{lstlisting}

\item Gere o controlador para os comments:
		\begin{lstlisting}[style=BashInputBasicStyle]
			$ rails generate controller comments
		\end{lstlisting}
		
\item Modifique a acão create do controlador \alert{controllers/comments\_controller.rb}:
\begin{lstlisting}[style=RubyInputStyle]
before_action :set_comment, only: [:show, :edit, :update, :destroy]

def create
  @post = Post.find(params[:post_id])
  @comment = @post.comments.create(comment_params)

	if @comment.save
  	redirect_to @post, notice: 'Comment foi criado com sucesso!' 
  else
    redirect_to @post
  end
end 

private
	def set_comment
  	@comment = Comment.find(params[:id])
  end

  def comment_params
    params.require(:comment).permit(:body)
  end
\end{lstlisting}

\item Escolha uma postagem qualquer e escreva alguns comentários.

%\item Remova a rota absoluta para comentários no arquivo de rotas e reinicie o servidor:
%\begin{lstlisting}[style=RubyInputStyle, caption=config/routes.rb]
%Rails.application.routes.draw do
%  #resources :comments
%  resources :posts do 
%    resources :comments
%  end 
%end 
%\end{lstlisting}	
	\end{itemize}
\end{frame}