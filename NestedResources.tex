\begin{frame}[allowframebreaks, t, fragile]{Hora de Colocar a Mão na Massa}
	\begin{itemize}
		\item Modifique o arquivo de rotas para aninhar os comentários às postagens e reinicie o servidor:
		\begin{lstlisting}[style=RubyInputStyle, caption=config/routes.rb]
Rails.application.routes.draw do
  resources :comments
  resources :posts do 
    resources :comments
  end 
end 
		\end{lstlisting}

		\item Crie um novo \verb|post| e salve no banco de dados:
		\begin{lstlisting}[style=BashInputBasicStyle]
			irb(main):005:0> p1 = Post.new
			irb(main):006:0> p1.title="Whatsapp bloqueado"
			irb(main):007:0> p1.body="A justiça bloqueou o Whatsapp..."
			irb(main):008:0> p1.save
		\end{lstlisting}
		
		\item Crie um novo \verb|comment| e o vincule a um \verb|post|:
		\begin{lstlisting}[style=BashInputBasicStyle]
			irb(main):005:0> c1 = Comment.new
			irb(main):006:0> c1.body="O que fazer agora ???"
			irb(main):007:0> c1.save
			irb(main):008:0> p1.comments << c1
		\end{lstlisting}
		
		\item Digite no navegador no endereço \url{http://localhost:3000/posts/id/comments}. Onde 
		o \alert{id} é o id do post criado anteriormente.
				
		\item Agora crie um novo blog e digite novamente \url{http://localhost:3000/posts/id/comments}. Onde 
		o \alert{id} do blog que acabou de ser criado.(\alert{Temos um problema})
				
		\item Modifique o código do template \alert{views/posts/show.html.erb}. Insira o código abaixo, após 
		a linhas 10 (abaixo do parágrafo do body).
		\begin{lstlisting}[style=RubyInputStyle]
<h2>Comments</h2>
<div id="comments">
  <% @post.comments.each do |comment| %>
    <%= div_for comment do %>
    <p>
      <strong>Posted <%= time_ago_in_words(comment.created_at) %></strong>
      <%= h(comment.body) %>
    </p>
    <% end %>
  <% end %>
</div>
		\end{lstlisting}

	\item Agora no navegador visualize uma postagem que tenha comentários.
	\item Acrescente o código a seguir logo abaixo do código anterior no arquivo \alert{views/posts/show.html.erb}:
	\begin{lstlisting}[style=RubyInputStyle]
<h2>Comments</h2>
  <div id="comments">
  <% @post.comments.each do |comment| %>
    <%= div_for comment do %>
    <p>
      <strong>Posted <%= time_ago_in_words(comment.created_at) %></strong>
      <%= h(comment.body) %>
    </p>
    <% end %>
  <% end %>
</div>
\end{lstlisting}

\item Modifique a acão create do controlador \alert{controllers/comments\_controller.rb}:
\begin{lstlisting}[style=RubyInputStyle]
def create
  @post = Post.find(params[:post_id])
  @comment = @post.comments.create(comment_params)

  respond_to do |format|
    if @comment.save
      format.html { redirect_to @post, notice: 'Comment was successfully created.' }
      format.json { render :show, status: :created, location: @comment }
    else
      format.html { render :new }
      format.json { render json: @comment.errors, status: :unprocessable_entity }
    end
  end
end 
\end{lstlisting}

\item Escolha uma postagem qualquer e escreva alguns comentários.

\item Remova a rota absoluta para comentários no arquivo de rotas e reinicie o servidor:
\begin{lstlisting}[style=RubyInputStyle, caption=config/routes.rb]
Rails.application.routes.draw do
  #resources :comments
  resources :posts do 
    resources :comments
  end 
end 
\end{lstlisting}	
	\end{itemize}
\end{frame}