
\begin{frame}[allowframebreaks, fragile,t]{Arrays}
  \begin{itemize}
    \item Coleção de objetos (auto-expandível)
    \item Indexado pelo operador (método) \alert{[ ]}
    \item Pode ser indexado por números negativos ou intervalos
    \item Tipos heterogêneos são permitidos em um mesmo array
    \item \%\{str1 str2\} pode ser utilizado para criar um array de strings
  \end{itemize}
  
  \lstinputlisting[style=RubyInputStyle, caption=arrays.rb]{codigos/ruby/07-arrays/arrays.rb}
\pagebreak
  \begin{itemize}
    \item Modificando arrays:
    \begin{itemize}
      \item criação: \alert{= [ ]}
      \item inclusão: \alert{push} ou {<<}
      \item remoção: \alert{pop} ou {shift}
    \end{itemize}
    \item Extração randômica de elementos com \alert{sample}
    \item Classificação ou inversão com \alert{sort!} ou {reverse!}
  \end{itemize}  
  
  \lstinputlisting[style=RubyInputStyle, caption=arrays2]{codigos/ruby/07-arrays/arrays2.rb}
  
\pagebreak
  \begin{itemize}
    \item Métodos úteis
    \begin{itemize}
      \item \alert{each} - percorre um array
      \item \alert{select} - filtra por seleção
      \item \alert{reject} - filtra por rejeição
      \item \alert{map} - modifica cada elemento do array
    \end{itemize}
  \end{itemize}  
  
  \lstinputlisting[style=RubyInputStyle, caption=arrays2]{codigos/ruby/07-arrays/array_processing.rb}

\end{frame}

\begin{frame}[fragile,t]{Recapitulando}
  \begin{itemize}
    \item A API de arrays é flexível e poderosa
    \item Existem diversas formas de processar um elemento do array
  \end{itemize}
\end{frame}



