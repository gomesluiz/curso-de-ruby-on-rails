
\begin{frame}[allowframebreaks, fragile,t]{Loops e Interações}
  \begin{center}
    \Large \verb!loop! \\ \verb!while e until! \\ \verb!for! \\ \verb!each e times!
  \end{center}   
\framebreak

  \begin{itemize}
	\item \verb!loop!
  \end{itemize}
  \lstinputlisting[style=RubyInputStyle, caption=loop.rb]{codigos/ruby/03-loop-e-interacoes/loop.rb}

\framebreak
  \begin{itemize}
	\item \verb!while e until!
  \end{itemize}
     
  \begin{columns}
    \begin{column}{0.5\textwidth}
  \lstinputlisting[style=RubyInputStyle, caption=while.rb]{codigos/ruby/03-loop-e-interacoes/while.rb}
    \end{column}
    \begin{column}{0.5\textwidth}  %%<--- here
  \lstinputlisting[style=RubyInputStyle, caption=until.rb]{codigos/ruby/03-loop-e-interacoes/until.rb}
    \end{column}
  \end{columns}

\framebreak
  \begin{itemize}
		\item \verb!for! (\alert{dificilmente empregado})
    \item \verb!each/times! é preferível
  \end{itemize}
  
  \lstinputlisting[style=RubyInputStyle, caption=for\_loop.rb]{codigos/ruby/03-loop-e-interacoes/for_loop.rb}

\framebreak
  \begin{itemize}
	\item \verb!each!
  \end{itemize}
     
  \begin{columns}
    \begin{column}{0.5\textwidth}
  \lstinputlisting[style=RubyInputStyle, caption=each\_1.rb]{codigos/ruby/03-loop-e-interacoes/each_1.rb}  
    \end{column}
    \begin{column}{0.5\textwidth} 
  \lstinputlisting[style=RubyInputStyle, caption=each\_2.rb]{codigos/ruby/03-loop-e-interacoes/each_2.rb}

    \end{column}
  \end{columns}
\end{frame}

%-------------------------------------------------------------------------------------- Início
\begin{frame}[allowframebreaks,fragile,t]{Exercícios}
  \begin{enumerate}
    \item Escreva um \textit{script} Ruby que sorteia um número de 1 a 10 e permite que 
    o usuário tente 3 vezes até acertá-lo. A cada tentativa errada, o programa informa
    se o número a adivinhar está abaixo ou acima. \textbf{Dica:} utilize rand(n) + 1 
  \end{enumerate}
\end{frame}

\begin{frame}[fragile,t]{Recapitulando}
  \begin{itemize}
    \item Existe muitas opções de loops e interações
    \item \verb!each! é \alert{preferível} ao loop \verb!for! para percorrer arrays
  \end{itemize}
\end{frame}



