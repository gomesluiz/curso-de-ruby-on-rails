
\begin{frame}[allowframebreaks, fragile,t]{Loops e Interações}
  \begin{itemize}
	\item \verb!loop!
  \end{itemize}
	\begin{lstlisting}[style=RubyInputStyle, caption=loop.rb]
i = 0
loop do
	i += 2
	puts i 
	if i == 10
		break
	end
end 
	\end{lstlisting}  

\framebreak
  \begin{itemize}
	\item \verb!while e until!
  \end{itemize}
     
  \begin{columns}
    \begin{column}{0.5\textwidth}
      \begin{lstlisting}[style=RubyInputStyle, caption=while.rb]
a = 10
while a > 9
	puts a 
	a -= 1 
end 
      \end{lstlisting}  
    \end{column}
    \begin{column}{0.5\textwidth}  %%<--- here
      \begin{lstlisting}[style=RubyInputStyle, caption=until.rb]
a = 9
until a >= 10 
  puts a                       
  a += 1                       
end 
      \end{lstlisting}
    \end{column}
  \end{columns}

\framebreak
  \begin{itemize}
		\item \verb!for! (\alert{dificilmente empregado})
    \item \verb!each/times! é preferível
  \end{itemize}
  
  \lstinputlisting[style=RubyInputStyle, caption=for\_loop.rb]{codigos/ruby/02-fluxo-de-controle/for_loop.rb}

\framebreak
  \begin{itemize}
	\item \verb!each!
  \end{itemize}
     
  \begin{columns}
    \begin{column}{0.5\textwidth}
      \begin{lstlisting}[style=RubyInputStyle, caption=each\_1.rb]
nomes = ['Joao','Maria','Ana']
nomes.each { |nome| puts nome }
      \end{lstlisting}  
    \end{column}
    \begin{column}{0.5\textwidth} 
      \begin{lstlisting}[style=RubyInputStyle, caption=each\_2.rb]
nomes = ['Joao','Maria','Ana']
n = 1 
nomes.each do |nome|
	puts "#{x}. #{nome}"
	n += 1
end
      \end{lstlisting}
    \end{column}
  \end{columns}
\end{frame}

%-------------------------------------------------------------------------------------- Início
\begin{frame}[allowframebreaks,fragile,t]{Exercícios}
  \begin{enumerate}
    \item Escreva um \textit{script} Ruby que sorteia um número de 1 a 10 e permite que 
    o usuário tente 3 vezes até acertá-lo. A cada tentativa errada, o programa informa
    se o número a adivinhar está abaixo ou acima. \textbf{Dica:} utilize rand(n) + 1 
  \end{enumerate}
\framebreak
  \begin{itemize}
    \item \textbf{Solução}:
  \end{itemize}	
  \lstinputlisting[style=RubyInputStyle, caption=exercicio-1-solucao.rb]{codigos/ruby/03-loop-e-interacoes/exercicio-1-solucao.rb}
\end{frame}

\begin{frame}[fragile,t]{Recapitulando}
  \begin{itemize}
    \item Existe muitas opções de loops e interações
    \item \verb!each! é \alert{preferível} ao loop \verb!for! para percorrer arrays
  \end{itemize}
\end{frame}



