%%-------------------------------------------------------------------------------------- Início
\section{Plataforma Ruby}
\begin{frame}[allowframebreaks,fragile,t]{Instalação do Ruby}
  \begin{enumerate}
    \item Abra o terminal de linha de comando do Linux teclando \verb!CTRL+ALT+T! 
    \item Instale algumas \alert{dependências} necessárias para que o interpretador da linguagem 
      possa ser executado e também para que ferramentas de apoio possam ser instaladas.  
    \begin{lstlisting}[style=BashInputStyle]
       $ sudo apt-get update 
       $ sudo apt-get install git-core curl zlib1g-dev 
	    build-essential libssl-dev libreadline-dev libyaml-dev 
	    libsqlite3-dev sqlite3 libxml2-dev libxslt1-dev 
	    libcurl4-openssl-dev python-software-properties
    \end{lstlisting}
    
    \framebreak
    \item Instale, agora, o interpretador e as bibliotecas usuais da linguagem Ruby:
    \begin{lstlisting}[style=BashInputStyle]
      $ sudo apt-get install libgdbm-dev 
	     libncurses5-dev automake libtool 
	     bison libffi-dev
      $ curl -L https://get.rvm.io | bash -s stable
      $ source ~/.rvm/scripts/rvm
      $ echo "source ~/.rvm/scripts/rvm" >> ~/.bashrc
      $ rvm install 2.1.2
      $ rvm use 2.1.2 --default
    \end{lstlisting}
    
    \item Teste se interpretador e as bibliotecas do Ruby foram instaladas corretamente, executando
      o seguinte comando:
    \begin{lstlisting}[style=BashInputStyle]
      $ ruby -v
    \end{lstlisting}
    
    \item A seguinte mensagem deverá ser apresentada se tudo foi instalado corretamente para
      o Ruby.
    \begin{lstlisting}[style=BashInputStyle]
      ruby 2.1.2p95 (2014-05-08 revision 45877) [i686-linux]
    \end{lstlisting}
    
    \item O último passo é configurar o gerenciador de pacotes do Ruby \verb!(Rubygens)! para
      que ele não instale a documentação de cada \alert{gem} localmente.
    \begin{lstlisting}[style=BashInputStyle]
      $ echo "gem: --no-ri --no-rdoc" > ~/.gemrc
    \end{lstlisting}
    
  \end{enumerate}
 
\end{frame}
