%%-------------------------------------------------------------------------------------- Início
\begin{frame}[allowframebreaks, t, fragile]{Ação: Update}
	\begin{itemize}
		\item Recupera um objeto \alert{post} utilizando o parâmetro \alert{id}	
		\item Atualiza o objeto \alert{post} com os parâmetros que foram passados pelo
			formulário \alert{edit}
		\item Tenta \alert{atualizar} o objeto no \alert{banco de dados}
		\item Se sucesso, redireciona para o template \alert{show}
		\item Se insucesso, renderiza o template \alert{edit} novamente
		\begin{lstlisting}[style=RubyInputStyle, caption=posts\_controller.rb]
Class PostsController < ApplicationController
  # PATCH/PUT /posts/1
  # PATCH/PUT /posts/1.json
  def update
    respond_to do |format|
      if @post.update(post_params)
        format.html { redirect_to @post, notice: 'Post was successfully updated.' }
        format.json { render :show, status: :ok, location: @post }
      else
        format.html { render :edit }
        format.json { render json: @post.errors, status: :unprocessable_entity }
      end
    end
  end
  private 
   # Never trust parameters from the scary internet, 
	 # only allow the white list through.
   def post_params 
     params.require(:post).permite(:title, :content)
   end 
		\end{lstlisting}				
	\end{itemize}	
\end{frame}


